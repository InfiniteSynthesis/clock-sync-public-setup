\subsection{Typical Executions}
\label{subsec:typical-exeuctions}

We define the notion of \emph{typical} executions following~\cite{C:GarKiaLeo17,EPRINT:GarKiaLeo20}.
%
The idea here is that given a certain execution $E$, we compare the actual progress and the expected progress that parties will make under the success probabilities.
%
If the difference and variance are reasonably small, and no bad events (see~\cref{def:insertion-copy-prediction}) about the underlying hash function happen, we declare $E$ \emph{typical}.

\begin{definition} \label{def:insertion-copy-prediction}
    An insertion occurs when, given a chain \chain with two consecutive blocks \block and $\block'$, a block $\block^*$ created after $\block'$ is such that $\block, \block^*, \block'$ form three consecutive blocks of a valid chain.
    %
    A copy occurs if the same block exists in two different positions.
    %
    A prediction occurs when a block extends one with later creation time.
\end{definition}

Note that in addition (compared to~\cite{C:GarKiaLeo17,EPRINT:GarKiaLeo20}), in~\cref{def:typical-execution}(a) we require that the difficulty of all blocks the alert parties can acquire during consecutive rounds $S$ (i.e., $D(S)$) is well lower-bounded.
%
This is because $D(S)$ also captures the beacon production process, where there is no loss incurred by the bounded-delay network as well as by skewed local clocks.
%
Hence, a reasonably better lower-bound on $D(S)$ helps us get better results when arguing for the good properties of generated beacons by alert parties (\cref{lemma:goodbeacons}).

\begin{definition}
    [Typical Execution]
    \label{def:typical-execution}

    An execution E is typical if the following hold.
    %
    \begin{enumerate}[label=(\alph*), leftmargin=*, noitemsep]
        \item For any set $S$ of at least $\ell$ consecutive good rounds,
              %
              \[ (1 - \epsilon)[1 - (1 + \delta) \gamma^2 f]^\delay p h(S) < Q(S) \le D(S) < (1 + \epsilon) p h(S) ~~\text{and}~~ D(S) > (1 - \epsilon) p h(S). \]

        \item For any set $J$ of consecutive adversarial queries and $\alpha(J) = 2(\frac{1}{\epsilon} + \frac{1}{3})\lambda / T(J)$,
              %
              \[ A(J) < p|J| + \max \{\epsilon p |J|, \tau \alpha(J)\} ~~\text{and}~~ B(J)< p|J| + \max \{\epsilon p |J|, \alpha(J)\}. \]

        \item No insertions, no copies, and no predictions occurred in $E$.
    \end{enumerate}
\end{definition}

In the next lemma, we establish the quantitative relation between honest and adversarial hashing power during consecutive rounds with length at least $\ell$, as well as the relationship between the total difficulty acquired by all parties ($D(S) + A(J)$) and their hashing power.

\begin{lemma} \label{lemma:typical-execution-result}
    Consider a typical execution in a $(\gamma, s)$-respecting environment.
    %
    Let $S = \{r : u \le r \le v \}$ be a set of at least $\ell$ consecutive good rounds and $J$ the set of adversarial queries in $U = \{ r : u - \delay - \maxskew \le r \le v + \delay + \maxskew \}$.
    %
    We have
    %
    \begin{enumerate}[label=(\alph*), leftmargin=*, noitemsep]
        \item $(1 + \epsilon) p|J| \le Q(S) \le D(U) < (1 + 5\epsilon) Q(S)$.

        \item $T(J) A(J) < \epsilon \diffLen f / 4(1 + \delta)$ or $A(J) < (1 + \epsilon)p|J|$; $\tau T(J) B(J) < \epsilon \diffLen f / 4(1 + \delta)$ or $B(J) < (1 + \epsilon) p|J|$.

        \item If $w$ is a good round such that $|w - r| \le s$ for any $r \in S$, then $Q(S) > (1 - \epsilon) [1 - (1 + \delta) \gamma^2 f]^\delay |S| p n_w / \gamma$.
              %
              If in addition $T(J) \ge T^{\min}_w$ , then $A(J) < (1 - \delta + 3\epsilon) Q(S)$.

        \item If $w$ is a good round such that $|w - r| \le s$ for any $r \in S$ and $T(J) \ge T^{\min}_w$, then $D(S) + A(J') < (1 + \epsilon) p(h(S) + |J'|)$ where $J'$ denotes the set of adversarial queries in $S$.
    \end{enumerate}
\end{lemma}

We conclude that almost all executions (that are polynomially bounded by $\kappa$ and $\lambda$) are typical.

\begin{theorem} \label{thm:typical-execution-error-prob}
    Assuming the ITM system $(\Z, C)$ runs for $L$ steps, the probability of the event ``\E is not typical'' is bounded by $O(L^2)(e^{-\lambda} + 2^{-\kappa})$.
\end{theorem}
