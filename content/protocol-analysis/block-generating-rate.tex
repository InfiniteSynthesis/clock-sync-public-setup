\subsection{Typical Executions Maintain a Steady Block Generation Rate}

We first show that the adversary cannot create a fork that happened too long ago compared with the current time.
%
We consider a novel partition strategy (different from the approach in~\cite{EPRINT:GarKiaLeo20}) that is better suited to the new target recalculation function.

\lemmanostalechains

\begin{proof}
    Suppose---towards a contradiction---$\chain \in \mathcal{S}_r$ and has not been extended by an honest party for at least $\ell + 2(\maxskew + \delay)$ rounds and $r$ is the least (nominal) round with this property.
    %
    Let \block be the last honestly-generated block of \chain (possibly the genesis) and let $w$ be the (nominal) round it was computed.
    %
    We consider $S = \{u : w + (\maxskew + \delay) \le u \le r - (\maxskew + \delay) \}$ and $U = \{u : w \le u \le r \}$ ($|S| \ge \ell$ by assumption).
    %
    Suppose that the blocks of \chain after \block (we denote these blocks by $B$) span $k$ epochs with corresponding targets $T_1, \ldots , T_k$.
    %
    For $i \in [k]$ let $m_i$ be the number of blocks with target $T_i$ and set $M = m_1 + \ldots + m_k$ and $d = m_1 / T_1 + \ldots + m_k / T_k$.
    %
    Our plan is to contradict the assumption that $C \in \mathcal{S}_r$ by showing that all chains in $\mathcal{S}_r$ have more difficulty than \chain.
    %
    By Chain-Growth \cref{lemma:chain-growth}, all the honest parties have advanced (in difficulty) during the rounds in $U$ by $Q(S)$.
    %
    Therefore, to reach a contradiction it suffices to show that $d < Q(S)$.

    Consider the following partition on $B$: we partition $B$ into $u$ sections $B_v, v \in [u]$ and associate each section $B_v$ with the target of its first block $T_v$.
    %
    Section $B_v$ starts with either the block after \block (if $v = 1$) or the $\lceil m_i / 2 \rceil$-th block in an epoch (if $v > 1$); it ends at either the last block of the chain (if $v = u$) or the $\lfloor m_i / 2 \rfloor$-th block such that in epoch $i + 1$ the target is less than $T_v / \tau$.
    %
    Under such construction, the next block after partition $B_v$ is exactly the first block of partition $B_{v + 1}$.

    For $u \ge 2$, we claim that for partition $B_v$, it has the following properties: (i) for all blocks in $B_v$, their target is at least $T_v / \tau$; (ii) the number of blocks in $B_v$ is at least $\diffLen f / 2$.

    Property (i) holds because of the strategy of our partition that will stop before it exceeds the lower bound for the targets and thanks to~\cref{eq:next-target} we need to pass at least two boundaries of epochs so the circumstance that no blocks exist in such partition will never happen.
    %
    To reason why property (ii) stands, consider those epochs that are split into two different sections.
    %
    For an epoch \epoch whose blocks are split into two sections $B_v, B_{v + 1}$, since in epoch $\epoch + 1$ the target is larger than that in \epoch (if not, it does not satisfy the criteria of the partition), there are at least $\diffLen f$ blocks in epoch \epoch.
    %
    Otherwise, \cref{eq:next-target} will raise the target.
    %
    By the rule of partition, at least $\diffLen f/ 2$ blocks are in each sections.
    %
    Hence for every partition, either its head or tail has at least $ \diffLen / 2$ blocks in the same epoch, and this implies the lower bound of the total number of blocks.

    For each $v \in \{1, 2, \ldots, u\}$, let $j_v \in J$ denote the index of the query during which the first block of the $v$-th section was computed and set $J_v = \{j : j_v \le j < j_{i + 1} \}$ (\cref{def:typical-execution}(c) assures $j_i < j_{i + 1}$).
    %
    We have
    %
    \[ d = \sum_{i = 1}^k \frac{m_i}{T_i} < \sum_{v = 1}^u (1 + \epsilon) |J_v| \le (1 + \epsilon) p |J| \le Q(S). \]
    %
    The difficulty of the blocks acquired in $J_v$ is at most $A(J_v)$ by property (i) and their number at most $T_v A(J_v)$.
    %
    Since property (ii) above shows that the adversary acquired at least $\diffLen / 2$ blocks in $J_v$, the desired bound follows from~\cref{lemma:typical-execution-result}(b).
    %
    The final inequality is~\cref{lemma:typical-execution-result}(a).

    If $u = 1$, let $J$ denote the queries in $U$ starting from the first adversarial query attempting to extend \block.
    %
    Then, $T_1 = T(J)$ and $T_i \ge T(J) / \tau (i > 1)$; thus, $d \le A(J)$.
    %
    If $A(J) < (1 + \epsilon) p |J|$, then $A(J) < Q(S)$ is obtained by~\cref{lemma:typical-execution-result}(a).
    %
    Otherwise, $A(J) < (\frac{1}{\epsilon} + 1) \tau \alpha(J) = 2(\frac{1}{\epsilon} + 1)(\frac{1}{\epsilon} + \frac{\epsilon}{3}) \tau \lambda / T(J)$.
    %
    However, we have
    %
    \begin{align*}
        Q(S) > (1 - \epsilon) & [1 - (1 + \delta) \gamma^2 f]^{\delay + \maxskew} \cdot \frac{p n_u \ell T_1}{\gamma T_1}
        \\
                              & > \frac{(1 - \epsilon)[1 - (1 + \delta) \gamma^2 f]^{\delay + \maxskew}  f \ell}{2 \gamma^3 T(J)} \ge \frac{2(1 - \epsilon)(1 + 3\epsilon) \tau \lambda}{\epsilon^2 T(J)} \ge A(J)
    \end{align*}
    %
    by considering only the first $\ell$ rounds in $S$ hence $n(S) \ge n_u \ell / \gamma$.
\end{proof}


\lemmacommonprefix

The proof is presented in~\cref{sec:proofs-omitted}.

Previous works~\cite{C:GarKiaLeo17,EPRINT:GarKiaLeo20} consider the common prefix parameters in terms of the number of blocks parties have to prune, which directly maps to the number of blocks produced in the time period implied by~\cref{lemma:commonprefix}.
%
With an imperfect local clock \funcImpClock as well as the new definition of common prefix (in terms of the number of rounds that we are going to remove), parties have to prune more rounds in order to guarantee it.
%
We establish the new Common Prefix parameter $k$ in~\cref{corollary:common-prefix-parameter}, which coincides with the \CPLen boundaries of \Isync{\cdot} in~\cref{eq:isync-def}, and will be helpful in order to argue the good properties of clock skews.

\begin{corollary} \label{corollary:common-prefix-parameter}
    For a typical execution in a $(\gamma, s)$-respecting environment, if all predicates in~\cref{def:analysis-predicate} hold till $r - 1$, for any two alert parties $\party_1, \party_2$ holding chains $\chain_1, \chain_2$ at round $r$, it holds that $\chain_1^{\lceil k} \preccurlyeq \chain_2$, where $k = \ell + 2\delay + 4\maxskew$.
\end{corollary}

\begin{proof}
    \cref{lemma:commonprefix} shows that common prefix can be acquired by pruning the blocks produced in the last $\ell + 2\delay + 2\maxskew$ nominal rounds.
    %
    We consider the timestamp of the first honest block \block produced during such nominal rounds.
    %
    $\textsc{GoodSkew}(r - 1)$ indicates that the timestamp of \block can be up to \maxskew-rounds behind, if \block is produced by an alert party that is \maxskew-rounds behind other alert parties.
    %
    Meanwhile, $\textsc{GoodShift}(r - 1)$ implies that up to \maxskew logical rounds may be skipped at the boundary of an interval.
    %
    Note that parties can pass at most two intervals within $\ell + 2\delay + 2\maxskew$ nominal rounds, and setting back local clocks does not hurt common prefix in that it is safe to prune more blocks.
    %
    By summing them up, we conclude $k = \ell + 2\delay + 4\maxskew$.
\end{proof}

\lemmablocklength

\begin{proof}
    Suppose---towards a contradiction---that $\textsc{BlockLength}(r)$ is false.
    %
    Then, there exists a $w \le r$ and a chain $\chain \in \mathcal{S}_w$ with an epoch of target $T$ and duration $\varLambda$ that does not satisfy
    %
    \[ \frac{1}{2(1 + \delta)\gamma^2} \cdot \diffLen f \le \varLambda \le 2(1 + \delta)\gamma^2 \cdot \diffLen f. \]
    %
    We consider the earliest epoch \epoch with this property.

    For the lower bound, we show that alert parties can alone produce enough blocks.
    %
    Let $u, v$ denote the first and last nominal round such that all alert parties are in epoch \epoch.
    %
    Such $u, v$ exist since $\textsc{GoodSkew}(r - 1)$ holds.
    %
    Define $S = \{i: u + \ell + 2\delay + 2\maxskew \le i \le v - (\ell + 2\delay + 2\maxskew) \}$.
    %
    We consider the length of $S$.
    %
    Since one epoch consists of 4 intervals and $\textsc{GoodShift}(r - 1)$, $\textsc{GoodSkew}(r - 1)$ holds, we get $v - u \ge \diffLen - 4\maxskew - 2\maxskew = \diffLen - 6\maxskew$ (we prune \maxskew rounds at the beginning and end to ensure that alert parties stay in \epoch).
    %
    Further, by applying Condition~\eqref{condition:min-length} we get $|S| \ge \diffLen - (2\ell + 4\delay + 10\maxskew) \ge (1 - \epsilon) \diffLen$.
    %
    We have
    %
    \[ Q(S) > (1 - \epsilon) [1 + (1 + \delta)\gamma^2 f]^{\delay + \maxskew} \cdot \frac{f |S|}{2\gamma^2 T} \ge (1 - \epsilon)^3 \cdot \frac{\diffLen f}{2\gamma^2 T} \ge \frac{\diffLen f}{2(1 + \delta) \gamma^2 } \cdot \frac{1}{T}. \]
    %
    The first inequality follows that nominal round $u - 1$ belongs to target recalculation zone $Z_\epoch$ and is good (note that $h_{u - 1} \ge h_w / \gamma, w \in S$ holds); the second and third inequalities comes from the lower bound on $|S|$ and Condition~\eqref{condition:absorb-error}.
    %
    By Chain Growth (\cref{lemma:chain-growth}), this implies that alert parties has produced at least $\diffLen f / 2(1 + \delta) \gamma^2$ blocks.

    For the upper bound, we show that even if the alert parties and the adversary join force, they cannot produce more than $2(1 + \delta)\gamma^2 \cdot \diffLen f$ blocks with timestamp in epoch \epoch.
    %
    Let $u, v$ denote the first and last nominal round such that at least one alert party is in epoch \epoch.
    %
    Define $S = \{ i : u \le i \le v \}$ and $S' = \{i : u - (\ell + 2\delay + 2\maxskew) \le i \le v + \ell + 2\delay + 2\maxskew \}$, and $J$ the set of queries available to the adversary during the rounds in $S'$ starting with the first query for target $T$ (so that $T(J) = T$).
    %
    $\textsc{CommonPrefix}(r)$ implies that all adversarial queries that contributed to epoch \epoch are all in $J$.
    %
    By $\textsc{GoodShift}(r - 1)$ and $\textsc{GoodSkew}(r - 1)$ we get $|S| \le \diffLen + 8\maxskew + 2\maxskew = \diffLen + 10\maxskew$ (we add \maxskew rounds at the beginning and end to ensure that at least one alert parties stay in \epoch).
    %
    Furthermore, $|S| < |S'| \le \diffLen + 2\ell + 4\delay + 14\maxskew \le (1 + \epsilon) \diffLen$ by Condition~\eqref{condition:min-length}.
    %
    Since $u - 1$ is a nominal round within target recalculation zone $Z_\epoch$ and $h_w \le \gamma h_{u - 1}$ for all $w \in S'$, it follows that $p h(S) \le p \gamma h_{u  -1} |S| \le (1 + \delta) \gamma^2 f|S| / T$.
    %
    Hence,
    %
    \[ D(S) < (1 + \epsilon) p h(S) \le (1 + \epsilon) (1 + \delta) \gamma^2 f |S| \cdot \frac{1}{T} < (1 + \epsilon)^2 \cdot  (1 + \delta) \gamma^2 \diffLen f \cdot \frac{1}{T}. \]

    Regarding the adversary \adv, if $\tau T B(J) < \epsilon \diffLen f / 4$, the total number of blocks is less than $2(1 + \delta) \gamma^2 \diffLen f$ and we are done.
    %
    Otherwise,
    %
    \[ B(J) < (1 + \epsilon) p |J| \le (1 + \epsilon) (1 - \delta) (1 + \delta) \gamma^2 f |S'| \cdot \frac{1}{T} \le (1 + \epsilon)^2 (1 - \delta) (1 + \delta)\gamma^2 \diffLen f \cdot \frac{1}{T}. \]
    %
    The second inequality holds because $t_w < (1 - \delta) h_w$ for all $w \in S'$, and the last one follows from the upper bound on $|S'|$.
    %
    Note that $\epsilon < \delta / 4$ (by Condition~\eqref{condition:absorb-error}), we have $(1 + \epsilon)^2 (2 - \delta) < 2$.
    %
    I.e., the total number of blocks is less than $2(1 + \delta)\gamma^2 \diffLen f$.
\end{proof}

Note that the length of $S'$ in the above proof implies that we should require $s \ge \diffLen + 2(\ell + 2\delay + 7\maxskew)$.

\lemmagoodchains

\begin{proof}
    Recall that the first target recalculation zone $Z_1$ (which consists of merely the first round) is good in our assumption, it suffices to show that if a recalculation zone $Z_\epoch$ is good, then the next one $Z_{\epoch + 1}$ is also good.
    %
    Let $\varLambda$ denote the number of blocks in epoch \epoch.
    %
    we wish to show that, for all $z \in Z_{\epoch + 1}$, $f / 2\gamma \le p h_z T_{\epoch + 1} \le (1 + \delta) \gamma f$.

    We first consider the lower bound.
    %
    Consider a round $w \in Z_\epoch$ and a round $z \in Z_{\epoch + 1}$.
    %
    If $\varLambda \le \diffLen f / \gamma$, we get $T_{\epoch + 1} \ge \gamma T_\epoch$ according to the target recalculation function (see~\cref{eq:next-target}).
    %
    Hence, $p h_z T_{\epoch + 1} \ge p h_w T_{\epoch + 1} / \gamma \ge p h_w T_\epoch \ge f / 2\gamma$ in that $w$ belongs a good recalculation zone.

    If not, assume $\varLambda > \diffLen f / \gamma$.
    %
    Let $u, v$ denote the first and last nominal round such that at least one alert party is in epoch \epoch and consider $S = \{i : u \le i \le v\}, S' = \{i : u - (\ell + 2\delay + 2\maxskew) \le i \le v + \ell + 2\delay + 2\maxskew \}$.
    %
    Let $J$ denote the set of queries available to the adversary in $S'$.
    %
    By~\cref{lemma:commonprefix}, all blocks contributed to $\varLambda$ were computed during honest queries in $S$ or adversarial ones in $J$.
    %
    From the discussion in~\cref{lemma:blocklength} we learn that $|S| < |S'| \le \diffLen + 4\delay + 14\maxskew \le (1 + \epsilon) \diffLen$.
    %
    In addition, for any round $z \in z_{\epoch + 1}$, $h(S') \ge h_z |S'| / \gamma$ in that $|S'| < s$; similarly, $h(S) \ge h_z |S| / \gamma$.
    %
    Thus, recall that $\diffLen = (T_{\epoch + 1} / T_\epoch) (1 / f) \varLambda$, we have
    %
    \[ B(J) < (1 - \delta)(1 + \epsilon) p h(S') \le (1 - \delta)(1 + \epsilon)^2 p \gamma h_u \diffLen \]
    %
    and $D(S) < (1 + \epsilon) p h(S) \le (1 + \epsilon)^2 p \gamma h_u \diffLen$.
    %
    For any $z \in Z_{\epoch + 1}$, assume $p h_z T_{\epoch + 1} < f / 2\gamma$, we get he following contradiction.
    %
    \[ 2 p \gamma h_z \diffLen = 2 p \gamma h_z \cdot \frac{T_{\epoch + 1}}{T_\epoch} \cdot \varLambda \cdot \frac{1}{f} < \varLambda \cdot \frac{1}{T_\epoch} \le D(S) + B(J) < (2 - \delta) (1 + \epsilon)^2 p \gamma h_z \diffLen < 2p \gamma h_z \diffLen. \]

    In order to prove the upper bound ($f \le (1 + \delta) \gamma f$), consider a round $w \in Z_\epoch$ and a round $z \in Z_{\epoch + 1}$.
    %
    If $\varLambda \ge \gamma \diffLen f $, we get $T_{\epoch + 1} \le T_\epoch / \gamma$.
    %
    Thus, $p h_z T_{\epoch + 1} \le p \gamma h_w T_{\epoch + 1} \le p h_w T_\epoch \le (1 + \delta) \gamma f$ in that $w$ belongs a good recalculation zone and we are done.

    Otherwise, assume $\varLambda < \gamma \diffLen f$.
    %
    Let $u, v$ denote the first and last nominal round such that all alert parties are in epoch \epoch and consider $S = \{i : u + \ell + 2\delay + 2\maxskew \le i \le v - (\ell + 2\delay + 2\maxskew) \}$.
    %
    Following the argument in~\cref{lemma:blocklength} we get $|S| \ge (1 - \epsilon) \diffLen$.
    %
    For any $t \in Z_{\epoch + 1}$, assume $p h_z T_{\epoch + 1} > (1 + \delta) \gamma f$ and recall that $\diffLen = (T_{\epoch + 1} / T_\epoch) (1 / f) \varLambda$, we obtain the following contradiction.
    %
    \[ \frac{p h_z \diffLen}{(1 + \delta) \gamma } = \frac{p h_z}{(1 + \delta) \gamma} \cdot \frac{T_{\epoch + 1}}{T_{\epoch}} \cdot \varLambda \cdot \frac{1}{f} > \varLambda \cdot \frac{1}{T_{\epoch}} \ge Q(S) \ge (1 - \epsilon) (1 - (1 + \delta) \gamma^2 f)^{\delay + \maxskew} \cdot \frac{p h_z |S|}{\gamma} \ge \frac{p h_z \diffLen}{(1 + \delta) \gamma}. \]
    %
    The second inequality comes from Chain Growth (\cref{lemma:chain-growth}) (recall that $\chain \in \mathcal{S}_r$ and the adversary can discard alert parties for at most $\ell + 2\delay + 2\maxskew$ rounds at the beginning and end of epoch as a result of $\textsc{CommonPrefix}(r)$, \cref{lemma:commonprefix}); the next one holds in that $\diffLen < s$ and hence $h(S) \ge h_z |S| / \gamma$; the last inequality follows Condition~\eqref{condition:absorb-error}.
\end{proof}

\corollarygoodrounds

\begin{proof}
    Consider any $\chain \in \mathcal{S}_r$.
    %
    Let $Z_\epoch$ be its last target recalculation zone before $r$.
    %
    If $r \in Z_\epoch$, it follows directly by~\cref{lemma:goodchains} that it is good.
    %
    Otherwise, consider a round $w \in Z_\epoch$ (recall that $f / 2\gamma \le p h_w T_{\epoch} \le (1 + \delta) \gamma f$).
    %
    Since $\textsc{GoodShift}(r - 1)$, $\textsc{GoodSkew}(r - 1)$ implies $r - w < \diffLen + 8\maxskew + \maxskew = \diffLen + 9\maxskew < s$, we have $h_r / \gamma \le h_w \le \gamma h_r$.
    %
    Combining these two bounds we obtain the desired inequality.
\end{proof}
