\section{Model and Building Blocks}
\label{sec:model}

In this paper we adapt the timing and networking model of~\cite{EC:BGKRZ21} to the setting of proof of work, obviating the requirement for a PKI as a setup assumption.
%
In more detail, in the model there is an upper bound \delay in message transmission (cf.~\cite{JACM:DwoLynSto88,EC:PasSeeash17,C:BMTZ17,EPRINT:GarKiaLeo20}), and parties do not have access to a global clock, but instead rely on their local clocks, whose drift is assumed to be upper-bounded by \maxdrift.
%
What complicates matters is that the model supports dynamic participation where parties may join and leave during the protocol execution without warning (it is worth noting here that this is where the difficulty of our setting is derived from: indeed if all honest parties were online throughout then it would be trivial to implement a logical clock by incrementing a counter).
%
For succinctness, we choose to express primitives and building blocks (see below) in our execution model utilizing the ideal functionality language of~\cite{FOCS:Canetti01}, but we do not pursue a composability analysis for our security properties, which are expressed in a game based manner as in~\cite{EC:GarKiaLeo15,EC:PasSeeash17}.

\subsection{Imperfect Local Clocks}
\label{subsec:imperfect-local-clock}

As in~\cite{EC:BGKRZ21}, and as mentioned above, in this paper we remove the assumption that parties have access to a global clock, as in~\cite{EC:GarKiaLeo15,C:GarKiaLeo17,EC:PasSeeash17,C:BMTZ17,EPRINT:GarKiaLeo20}, and instead assume \emph{imperfect local clocks}.
%
In a nutshell, every honest party maintains a local clock variable by communicating with an imperfect local clock functionality \funcImpClock.
%
In contrast to the global-setup clock functionality in~\cite{TCC:KMTZ13}, where parties learn the exact global time and thus strong synchrony is guaranteed, parties registered with \funcImpClock will only receive ``ticks'' from the functionality to indicate that they should update their own clocks.
%
In addition, \funcImpClock issues ``imperfect'' ticks, i.e., the adversary is allowed to set a bounded additive drift to each party by manipulating its corresponding status variable in \funcImpClock.
%
\funcImpClock can be viewed as a variant from~\cite{EC:BGKRZ21}'s with adaptations to provide a more natural clock model with real-word resources and in the proof-of-work setting.

For a detailed description of the functionality, see~\cref{section:model-contd}.
%
Here we just elaborate on the ``imperfect'' aspect of the clock and on the adversarial manipulation of clock drifts.
%
Specifically, we allow the adversary to set some drifts to parties' local clocks, which will accelerate or stall their progress; such values are globally bounded by \maxdrift.
%
This assumption allows local clocks to proceed at ``roughly'' the same speed.

Further, the adversary \adv can adaptively manipulate the drift of honest parties' clocks by sending \textsc{clock-forward} and \textsc{clock-backward} messages to the functionality\footnote{As such, our clock functionality is a more natural model of the real world compared to~\cite{EC:BGKRZ21}'s, as it allows \adv to manipulate the clock in both directions, backward, and forward; in~\cite{EC:BGKRZ21}, only forward manipulation is allowed. Nonetheless, this does not result in a more powerful adversary.} after they conclude the current round.
%
If \adv issues \textsc{clock-forward} for party \party, it will enter a new local round before \funcImpClock updates the nominal time, and this can be repeated as long as \party's drift is not \maxdrift rounds larger than other honest parties.
%
On the other hand, if \adv issues \textsc{clock-backward}, it will set \party's budget to a negative value, thus preventing \funcImpClock from updating $d_\party$ at the end of the nominal round ($d_\party$ is the functionality variable that captures whether the party \party has made its move for the clock tick).
%
I.e., \party will still be in the same logical round during these two nominal rounds.
%
Again, this process can be repeated by \adv as long as the drift on \party is not \maxdrift rounds smaller than others.
%
As a consequence, the targeted party's local clock may remain static for several nominal rounds.

\subsection{Other Core Functionalities}
\label{subsec:core-functionalities}

\paragraph{Common Reference String.}
%
We model a public-state setup by the CRS functionality \funcCRS.
%
The functionality is parameterized with some distribution $\mathcal{D}$ with sufficiently high entropy.
%
Once \funcCRS receives $(\textsc{Retrieve}, \sid)$ from either the adversary \adv or a party \party for the first time, it generates a string $d \gets \mathcal{D}$ as the common reference string.
%
In addition, \funcCRS will immediately send a message $(\textsc{Retrieved}, \sid)$ to functionality \wrapper{\funcRO} (described next) to indicate that \wrapper{\funcRO} should start to limit the adversarial RO queries.
%
For all subsequent activations, \funcCRS simply returns $d$ to the requester.

\paragraph{(Wrapped) Random Oracle.}
%
By convention, we model parties' calls to the hash function used to generated proofs of work as assuming access to a random oracle; this is captured by the functionality \funcRO.
%
Notice that with regards to bounding access to real-world resources, functionality \funcRO as defined fails to limit the adversary on making a certain number of queries per round.
%
Hence, we adopt a functionality wrapper \cite{C:BMTZ17,EC:GKOPZ20} \wrapper{\funcRO} that wraps the corresponding resource to capture such restrictions.
%
We highlight that our wrapper \wrapper{\funcRO} improves on previous wrappers in two aspects, in order to provide a more natural model of the real world:
%
(i) We capture the pre-mining stage by letting the adversary query the RO with no restrictions (albeit polynomially bounded) before the CRS is released;
%
and (ii) The wrapper limits adversarial access per nominal round by bounding the total number of queries that \adv can make.
%
The second aspect allows us to dispose the ``flat'' computational model and define the computational power in terms of the number of RO queries per round, which makes it possible to further refine the notion of a ``respecting environment'' (see below) that is suited for imperfect local clocks.

\paragraph{Diffusion network.}
%
We adopt the peer-to-peer communication functionality \funcDiffuse (cf.~\cite{EC:BGKRZ21}), which guarantees that an honestly sent message will be delivered to all the protocol participants within \delay rounds.
%
Moreover, for those adversarially generated messages, $\funcDiffuse^\delay$ forces them to be delivered to all the honest parties within $\delay$ rounds after they are learnt by at least one honest participant.
%
This captures the natural behavior of honest parties that they will forward all the messages that they have not yet seen to their peers.

\medskip
%
We refer to~\cref{section:model-contd} for a detailed description of the above functionalities.

\subsection{Dynamic Participation}
\label{subsec:dynamic-participation}

The notion of a ``respecting environment'' was introduced in~\cite{C:GarKiaLeo17} to model the varying number of participants in a protocol execution.
%
In~\cite{CCS:BGKRZ18,EC:BGKRZ21}, the notion of \emph{dynamic participation} was introduced aiming at describing the protocol execution in a more realistic fashion.
%
Here we present a further refined classification of possible \emph{types} of honest parties.
%
See~\cref{table:dynamic-participation}.

\begin{tabularx}{\textwidth}{X c c}
    \toprule
     & \multicolumn{2}{c}{\textbf{Basic types of \textit{honest} parties}}
    \\
    \textbf{Resource}
     & \textbf{Resource unavailable}
     & \textbf{Resource available}
    \\ \midrule
    random oracle $\funcRO$
     & \emph{stalled}
     & \emph{operational}
    \\
    network \funcDiffuse
     & \emph{offline}
     & \emph{online}
    \\
    clock \funcImpClock
     & \emph{time-unaware}
     & \emph{time-aware}
    \\
    synchronized state
     & \emph{desynchronized}
     & \emph{synchronized}
    \\ \bottomrule
    \caption{A classification of protocol participants.}
    \label{table:dynamic-participation}
\end{tabularx}

Consider an honest party \party at a given point of the protocol execution.
%
We say \party is \emph{operational} if \party is registered with the random oracle \funcRO; otherwise, we say it is \emph{stalled}.
%
We say \party is \emph{online} if \party is registered with the network; \emph{offline} otherwise.
%
We say \party is \emph{time-aware} if \party is registered with the \emph{imperfect} clock functionality \funcImpClock, and \emph{time-unaware} otherwise.

Further, we say \party is \emph{synchronized} if \party has been participating in the protocol for sufficiently long time and achieves a ``synchronized state'' as well as a ``synchronized clock.''
%
``Synchronized clock'' means \party holds a chain that shares a common prefix (cf.~\cite{EC:GarKiaLeo15}) with other \emph{synchronized} parties; ``synchronized clock'' refers to that \party maintains a local clock with time close to other \emph{synchronized} parties.
%
Otherwise, \party is \emph{desynchronized}.
%
Additionally, \party is aware of whether it is synchronized or not, and maintains a local variable \isSync serving as an indicator for other actions.

Based on the above classification, we now define the notion of \emph{alert} parties:
%
\[ alert \defeq operational \wedge online \wedge time\text{-}aware  \wedge synchronized. \]
%
In short, alert parties are those who have access to all the resources and are synchronized; this requires them to join the protocol execution passively for some period of time.
%
They constitute the core set of parties that carry out the protocol.

In addition, we define \emph{active} parties to include all parties that are alert, adversarial, and time-unaware.
%
\[ active \defeq alert \vee adversarial \vee time\text{-}unaware. \]

\paragraph{Respecting environment in terms of computational power.}
%
Next, we provide the following generalization of ``respecting environment'' to relate it to computational power as opposed to number of parties.
%
Our assumption is that during the whole protocol execution, the honest computational power is higher than the adversarial one (cf. the ``honest majority'' condition in~\cite{EC:GarKiaLeo15} and follow-ups).
%
The computational power is captured by counting the number of RO (hash) queries that parties make in each round.
%
Further, we restrict the environment to fluctuate the number of such queries in a certain limited fashion.

\begin{definition} \label{def:respecting-env}
    For $\gamma \in \mathbb{R}^+$ we call the sequence $(h_r)_{r \in [0, B)}$, where $B \in \mathbb{N}$, $(\gamma, s)$-respecting if for any set $S \subseteq [0, B)$ of at most $s$ consecutive integers, $\max_{r \in S} h_r \le \gamma \cdot \min_{r \in S} h_r$.
\end{definition}

We say that \emph{environment \Z is $(\gamma, s)$-respecting} if for all \adv and coins for \Z and \adv the sequence of honest hash queries $(h_r)$ is $(\gamma, s)$-respecting.

Note that the notion of respecting environment here is different from the ``flat'' model adopted in \cite{EC:GarKiaLeo15,C:GarKiaLeo17,EPRINT:GarKiaLeo20,C:BMTZ17}.
%
In a flat model, honest parties are assumed to have the same computational power, hence the total number of RO queries is a direct 1-to-1 map from the number of parties.
%
The new respecting environment allows some subset of the honest parties to query the RO multiple times or stay stalled during a nominal round and hence it adapts to the ``imperfect local clock'' model used in this paper.

