\section{Model and Building Blocks (Cont'd)}
\label{section:model-contd}

\paragraph{Imperfect local clock functionality.}
%
Functionality \funcImpClock maintains an internal variable $\tau_\sid$ (the nominal time) to record how many times the sub-procedure \textit{Round-Update} resets the round status for suitable honest parties and other functionalities.
%
This variable will never be revealed to neither honest parties nor registered
functionalities, which captures the fact that clocks are local, and it is only
accessible by \funcImpClock, the environment \Z, the adversary \adv and the (wrapped) random oracle functionality\wrapper{\funcRO}, after they send a \textsc{clock-read} command.
%
Notably, \wrapper{\funcRO} has to know the exact nominal time in order to coordinate with the environment so as to apply restrictions on adversarial queries and model dynamic participation.
%
(More details on the wrapped RO functionality and the notion of a respecting environment below.)

\funcImpClock also keeps track of the set of honest parties \partyset and the registered functionalities $F$.
%
For all honest parties and functionalities, \funcImpClock associates each of them with a bit variable $d \in \{0, 1\}$ which tracks if the corresponding party/functionality has claimed finishing its local round, and returns a response to command \textsc{clock-tick} to indicate whether the requester should update its local clock.
%
Importantly, every honest party \party is associated with a (possibly negative) integer $b_\party$ which indicates the ``tick budget'' allocated for \party during the current nominal round.
%
If no adversarial clock drift is set, at the end of a nominal round $b_\party$ is reset to 1 by \textit{Round-Update} and consumed by the party when \party sends \textsc{clock-update} in the next nominal round.
%
This captures the fact that, without adversarial manipulation, local clocks proceed with exactly the same speed.
%
Additionally, a (possibly negative) integer $\mathtt{drift}$ is associated with \party to record the total drift that is applied to its local clock.

An honest party \party sends the \textsc{clock-update} command to \funcImpClock to indicate conclusion of its current local round.
%
The main purpose of this command is to set the variable $d$ associated with \party to 1 and consume one tick from its budget (i.e., $b_\party \gets b_\party - 1$).
%
When party \party is activated, it sends $(\textsc{clock-tick}, \sid)$ to \funcImpClock to check if its corresponding indicator $d$ has been reset to 0.
%
If this happens, \party updates its local clock and enters the next local round; otherwise, this implies that \party need to wait for other participants.

Regarding the \textit{Round-Update} procedure, \funcImpClock will update the nominal time only when (i) all honest parties and registered functionalities have claimed finishing the current round (i.e., the corresponding indicator $d = 1$), and (ii) all the tick budgets allocated to honest parties have been consumed (i.e., $b_\party \le 0$ for every party).
%
When the party/functionality statuses pass the above checks, \textit{Round-Update} moves the nominal time forward ($\tau_\sid \gets \tau_\sid + 1$).
%
Then, \textit{Round-Update} issues a new tick budget for all honest parties (i.e., all budget variables $b$ are increased by 1); afterwards, for all registered functionalities and honest parties whose $b_\party$ is positive, their $d$ indicators are reset to 0.

Next, we elaborate on the ``imperfect'' aspect of the clock and on the adversarial manipulation of clock drifts.
%
Specifically, in addition to \textit{Round-Update}, we also allow the adversary to set some drifts to parties' local clocks, which will accelerate or stall their local clock; such values are globally bounded by \maxdrift.
%
This assumption allows local clocks to proceed at ``roughly'' the same speed.

Further, the adversary \adv can adaptively manipulate the drift of honest parties' clocks by sending \textsc{clock-forward} and \textsc{clock-backward}\footnote{As such, our clock functionality is a more natural model of the real world compared to \cite{EC:BGKRZ21}'s, as it allows \adv to manipulate the clock in both directions, backward, and forward; in \cite{EC:BGKRZ21}, only forward manipulation is allowed. Nonetheless, this does not result in a more powerful adversary.} after they conclude the current round.
%
If \adv issues \textsc{clock-forward} for party \party, it will enter a new local round before \funcImpClock updates the nominal time, and this can be repeated as long as \party's drift is not \maxdrift rounds larger than other honest parties.
%
On the other hand, if \adv issues \textsc{clock-backward}, it will set \party's budget to a negative value, thus preventing \funcImpClock from updating $d_\party$ at the end of the nominal round.
%
I.e., \party will still be in the same logical round during these two nominal rounds.
%
Again, this process can be repeated by \adv as long as the drift on \party is not \maxdrift rounds smaller than others.
%
As a consequence, the targeted party's local clock may remain static for several nominal rounds.

\begin{cccFunctionality}
    {\funcImpClock}
    {imperfect-local-clock}
    {The imperfect local clock.}

    This functionality maintains state variables as follows.

    \begin{minipage}{\linewidth}
        \addtocounter{table}{-1}
        \begin{tabularx}{.9\textwidth}{c X}
            \toprule[.3mm]
            \textbf{State Variable}
             & \textbf{Description}
            \\ \midrule[.3mm]
            $\partyset \gets \emptyset$
             & The set of registered parties $\party = (\pid, \sid)$.
            \\ \midrule
            $F \gets \emptyset$
             & The set of registered functionalities (together with their session identifier).
            \\ \midrule
            $\tau_\sid \gets 0$
             & The nominal-time variable for session \sid.
            \\ \midrule
            $d_\party \gets 0$
             & The clock-update variable for $\party = (\pid, \sid) \in \partyset$.  $d_\party$ is set to $1$ after \party finishes a round.
            \\ \midrule
            $b_\party \gets 1$
             & The tick-budget variable for $\party = (\pid, \sid) \in \partyset$.
            \\ \midrule
            $\mathtt{drift}_\party \gets 0$
             & The accumulated additive drifts for $\party = (\pid, \sid) \in \partyset$.
            \\ \midrule
            $d(\F, \sid) \gets 0$
             & The clock-update variable for $(\F, \sid) \in F$.
            \\ \bottomrule[.3mm]
        \end{tabularx}
    \end{minipage}

    \medskip\paragraph{Clock capabilities:}
    %
    \begin{cccItemize}[nosep]
        \item Upon receiving $(\textsc{clock-update}, \sid_C)$ from some party $\party \in \partyset$ set $d_\party \gets 1$ and $b_\party \gets b_\party - 1$; execute \textit{Time-Update} and forward $(\textsc{clock-update}, \sid_C , \party)$ to \adv.

        \item Upon receiving $(\textsc{clock-update}, \sid_C)$ from some functionality \F in a session \sid such that $(\F, \sid) \in F$ set $d(\F, \sid) \gets 1$, execute \emph{Time-Update} and return $(\textsc{clock-update},\allowbreak \sid_C, \F)$ to this instance of \F.

        \item Upon receiving $(\textsc{clock-forward}, \sid_C , \party = (\pid, \sid))$ from \adv where $\party \in \partyset$, if $d_\party = 0$ or $\mathtt{drift}_\party \ge \maxdrift$ or $\mathtt{drift}_\party - \min_{\party' = (\cdot, \sid)} \{ \mathtt{drift}_{\party'} \} \ge \maxdrift$, ignore the message.
        %
        Otherwise, update $d_\party = 0$, $b_\party \gets b_\party + 1$ and $\mathtt{drift}_\party \gets \mathtt{drift}_\party + 1$; return $(\textsc{clock-forward-ok}, \sid_C , \party)$ to \adv.

        \item Upon receiving $(\textsc{clock-backward}, \sid_C , \party)$ from \adv where $\party \in \partyset$, if $d_\party = 0$ or$\mathtt{drift}_\party \le - \maxdrift$ or $\max_{\party' = (\cdot, \sid)} \{ \mathtt{drift}_{\party'} \} - \mathtt{drift}_\party \ge \maxdrift$, ignore the message.
        %
        Otherwise, update $b_\party \gets b_\party - 1$ and $\mathtt{drift}_\party \gets \mathtt{drift}_\party - 1$; return $(\textsc{clock-backward-ok}, \sid_C , \party)$ to \adv.

        \item Upon receiving $(\textsc{clock-tick}, \sid_C)$ from any participant \party --- including the environment on behalf of a party --- or the adversary on behalf of a corrupted party \party (resp. from any ideal---shared or local---functionality \F), execute procedure \textit{Time-Update}, return $(\textsc{clock-tick}, \sid_C, d_{\party})$ (resp. $(\textsc{clock-tick}, \sid_C, d_{(\F, \sid)})$) to the requestor (where \sid is the session id of the calling instance).

        \item Upon receiving $(\textsc{clock-read}, \sid_C)$ from the adversary or the wrapper functionalities, return $(\textsc{clock-read},\sid_C, \tau_{\sid})$ to the requestor (where \sid is the session id of the calling instance).
    \end{cccItemize}

    \smallskip\emph{Procedure Time-Update:}
    %
    For each session \sid do: If (i) $d_{(\func, \sid)} = 1$ for all $\func \in F$, and (ii) $d_\party = 1$ and $b_\party \le 0$ for all honest parties $\party = (\cdot, \sid) \in \partyset$, then update $\tau_\sid \gets \tau_\sid + 1$, $d_{(\func,\sid)} \gets 0$ and $b_\party \gets b_\party + 1$ for all parties $\party = (\cdot, \sid) \in \partyset$.
    %
    Additionally, for all parties $\party = (\cdot, \sid) \in \partyset$ with $b_\party > 0$, update $d_\party \gets 0$.
\end{cccFunctionality}


\paragraph{Global random oracle functionality and its wrapper.}
%
By convention, we model parties' calls to the hash function used to generated proofs of work as assuming access to a random oracle; this is captured by the functionality \funcRO.
%
\funcRO internally maintains an updatable table $H$; it is parameterized with security parameter $\kappa$ as $H$'s output length.
%
Upon receiving a query $(\textsc{eval}, \sid, x)$, if $H(x) = \bot$ (i.e., no pair of the form $(x, \cdot)$ is in $H$), a value $y$ is chosen uniformly at random from $\{0, 1\}^\kappa$ and returned to the party (\funcRO also updates $H(x) = y$).
%
If $H(x) \neq \bot$ (i.e., $x$ has been queried before), the corresponding $y$ is returned.

\begin{cccFunctionality}{\funcRO}
    {random-oracle}
    {The random oracle.}

    The functionality is parameterized by a security parameter $\kappa$.

    \addtocounter{table}{-1}
    \begin{tabularx}{.9\textwidth}{c  X}
        \toprule[.3mm]
        \textbf{State Variable}
         & \textbf{Description}
        \\ \midrule[.3mm]
        $\partyset \gets \emptyset$
         & The set of registered parties.
        \\ \midrule
        $H \gets \emptyset$
         & A dynamically updatable function table where $H[x] = \bot$ denotes the fact that no pair of the form $(x, \cdot)$ is in $H$.
        \\ \bottomrule[.3mm]
    \end{tabularx}

    \begin{cccItemize}[noitemsep]
        \item \textbf{Eval.} Upon receiving $(\textsc{eval}, \sid, x)$ from some party $\party \in \partyset$ (or from \adv on behalf of a corrupted \party), do the following:
        %
        \begin{cccEnum}[nosep]
            \item If $H[x] = \bot$ sample a value $y$ uniformly at random from $\{0, 1\}^\kappa$ and set $H[x] \gets y$.
            \item Return $(\textsc{eval}, \sid, x, H[x])$ to the requestor.
        \end{cccEnum}
    \end{cccItemize}
\end{cccFunctionality}


Notice that with regards to bounding access to real-world resources, functionality \funcRO as defined fails to limit the adversary on making a certain number of queries per round.
%
Hence, we adopt a functionality wrapper \cite{C:BMTZ17,EC:GKOPZ20} \wrapper{\funcRO} that wraps the corresponding resource to capture such restrictions.
%o
We highlight that our wrapper \wrapper{\funcRO} improves on previous wrappers in two aspects, in order to provide a more natural model of the real world:
%
\begin{cccItemize}[noitemsep]
      \item We capture the pre-mining stage by letting the adversary query the RO with no restrictions before the CRS is released.
      %
      More specifically, \wrapper{\funcRO} is initialized with internal time counter $\tau = 0$, and the adversary can make as many queries as he wants (albeit polynomially bounded) and the wrapper simply forwards all these queries to \funcRO as long as $\tau = 0$.
      %
      This pre-mining stage ends when the time counter $\tau$ is set to $1$, which happens immediately after \wrapper{\funcRO} receives $(\textsc{Retrieved}, \sid)$ from the CRS functionality \funcCRS.

      This pre-mining stage captures the fact that a hash function is usually available before the protocol execution begins.
      %
      Nonetheless, thanks to the CRS with sufficiently high entropy, all adversarial queries made during the pre-mining stage can benefit the adversary in the later execution only with negligible probability.

      \item The wrapper limits adversarial access per nominal round by bounding the total number of queries that \adv can make.
      %
      After the CRS is released, \wrapper{\funcRO} keeps track of the corrupted party set.
      %
      Upon receiving queries from corrupted parties, \wrapper{\funcRO} first checks if \funcImpClock advances the nominal time (\wrapper{\funcRO} can directly access \funcImpClock's internal time counter).
      %
      If \funcImpClock enters the next round, \wrapper{\funcRO} updates its internal clock counter as well and resets the RO counter $h_\tau$
      to 0.
      %
      As long as $q_\adv < h_\tau$, where $h_\tau$ is a predetermined value provided by the environment \Z for nominal round $\tau$, \wrapper{\funcRO} forwards the queries to \funcRO.
      %
      When $q_\adv \ge h_\tau$, which implies that the adversary has
      consumed all its allowed budget, the wrapper stops interaction with
      the adversary.

      This ``migration'' from the $q$-bounded adversarial model (\cite{EC:GarKiaLeo15} and follow-ups) to bounding the total number of queries that \adv can make allows us to dispose of the ``flat'' computational model considered in previous works.
      %
      We now define the computational power in terms of the number of RO queries per round, which makes it possible to further refine the notion of a ``respecting environment'' that is suited for imperfect local clocks.
\end{cccItemize}

\begin{cccFunctionality}
    {\wrapper{\funcRO}}
    {random-oracle-wrapper}
    {The random oracle wrapper.}

    This functionality maintains state variables as follows.

    % \begin{minipage}{\linewidth}
    \begin{tabularx}{.9\textwidth}{c  X}
        \toprule[.3mm]
        \textbf{State Variable}
         & \textbf{Description}
        \\ \midrule[.3mm]
        $\partyset \gets \emptyset$
         & The set of registered parties; the current set of corrupted parties is denoted by $\partyset'$.
        \\ \midrule
        $\tau \gets 0$
         & The (real-time) clock tick counter.
        \\ \midrule
        $h_\tau$
         & An upper bound which restricts the \func-evaluations of all alert parties at time $\tau$.
        \\ \midrule
        $q_\honestPartySet, q_\adv \gets 0$
         & The alert/adversary evaluation counter.
        \\ \bottomrule[.3mm]
    \end{tabularx}
    \addtocounter{table}{-1}
    % \end{minipage}

    % \medskip
    \paragraph{Pre-mining attack handling (executed only if $\tau = 0$):}
    %
    \begin{cccItemize}[nosep]
        \item Upon receiving $(\textsc{eval}, \sid, x)$ from \adv on behalf of a corrupted party $P \in \partyset'$, forward the request to \funcRO and return to \adv whatever \funcRO returns.

        \item Upon receiving $(\textsc{Retrieved}, \sid)$ from \funcCRS, set $\tau = 1$.
    \end{cccItemize}

    \paragraph{Relaying inputs to the random oracle:}
    %
    \begin{cccItemize}[nosep]
        \item Upon receiving $(\textsc{eval}, \sid, x)$ from \adv on behalf of a corrupted party $P \in \partyset'$ or a de-synchronized party \party, first execute \textit{Round Reset}, then do the following.
        %
        \begin{cccEnum}[nosep]
            \item Set $q_\adv \gets q_\adv + 1$.
            \item \textbf{If} $q_\adv < h_\tau$ \textbf{then} forward the request to \funcRO and return to \adv whatever \funcRO returns.
        \end{cccEnum}

        \item Upon receiving $(\textsc{eval}, \sid, x)$ from an alert party \party, first execute \textit{Round Reset}, then do the following.
        %
        \begin{cccEnum}[nosep]
            \item Set $q_\honestPartySet \gets q_\honestPartySet + 1$.
            \item \textbf{If} $q_\honestPartySet \le h_\tau$ \textbf{then} forward the request to \funcRO and return to \party whatever \funcRO returns.
            \item \textbf{If} $q_\honestPartySet \ge h_\tau$ \textbf{then} send $(\textsc{clock-update}, \sid_C)$ to \funcImpClock.
        \end{cccEnum}
    \end{cccItemize}

    \paragraph{Corruption handling:}
    %
    \begin{cccItemize}[nosep]
        \item Upon receiving $(\textsc{corrupt}, \sid, \party)$ from the adversary, set $\partyset' \gets \partyset' \cup \party$.
    \end{cccItemize}

    \medskip\emph{Procedure Round-Reset:}
    %
    Send $(\textsc{clock-read}, \sid_C)$ to \funcImpClock and receive $(\textsc{clock-read}, \allowbreak \sid_C, \tau')$ from \funcImpClock. If $|\tau - \tau' | > 0$, then do the following.
    %[
    \begin{cccEnum}[nosep]
        \item Set $q_\honestPartySet, q_\adv \gets 0$ and $\tau \gets \tau'$.

        \item Send $(\textsc{next-round})$ to \adv and receive as response $(\textsc{next-round}, h^*_{\tau'})$.
        %
        \textbf{If} $(h_1, h_2, \ldots, h^*_{\tau'})$ is $(\gamma, s)$-respecting \textbf{then} set $h_{\tau'} = h^*_{\tau'}$; \textbf{else} set $h_{\tau'} = h_{\tau' - 1}$.
    \end{cccEnum}
\end{cccFunctionality}



\paragraph{Common reference string.}
%
We model a public setup by the CRS functionality \funcCRS.

\begin{cccFunctionality}
    {\funcCRS}
    {CRS}
    {The common reference string.}

    The functionality is parameterized by a distribution $\mathcal{D}$.

    \begin{cccItemize}[noitemsep]
        \item \textbf{Retrieve.} Upon receiving $(\textsc{Retrieve}, \sid)$ from some party \party (or from \adv on behalf of a corrupted \party), do the following:
        %
        \begin{cccEnum}[nosep]
            \item If activated for the first time, choose a value $d \gets \mathcal{D}$, and send $(\textsc{Retrieved}, \sid)$ to \wrapper{\funcRO}.

            \item Return $(\textsc{Retrieve}, d)$ to \party.
        \end{cccEnum}
    \end{cccItemize}
\end{cccFunctionality}

\paragraph{Diffusion functionality.}
%
There are three types of messages that are exchanged in the network: blockchain messages (e.g., when a party mines a new block); regular messages (usually transactions), which are diffused to the network when received by the environment; and synchronization beacons---a special message that parties will use to perform the local time adjustment.
%
For simplicity, we follow the convention that each type of messages is diffused by its own network.
%
We denote the network functionality disseminating blockchain information by $\funcDiffuse^{\mathsf{bc}}$; the one diffusing transactions by $\funcDiffuse^{\mathsf{tx}}$; and the one circulating synchronization beacons by $\funcDiffuse^{\mathsf{sync}}$, respectively.
%
Parties will communicate with $\funcDiffuse^{\mathsf{bc}}$ and $\funcDiffuse^{\mathsf{tx}}$ to receive new chains and transactions in \textsf{FetchInformation}, and they will receive synchronization beacons by sending requests to $\funcDiffuse^{\mathsf{sync}}$ in \textsf{ProcessBeacons}.
%
Refer to~\cref{protocol:fetch-information,protocol:process-beacons} for further details.

\begin{cccFunctionality}
      {\funcDiffuse}
      {diffuse}
      {The diffusion network.}

      \newcommand*{\msgid}{\ensuremath{\mathsf{mid}}\xspace}
      \newcommand*{\vecM}{\ensuremath{\vec{M}}\xspace}

      This functionality maintains state variables as follows.

      \begin{minipage}{\linewidth}
            \begin{tabularx}{.9\textwidth}{c  X}
                  \toprule[.3mm]
                  \textbf{State Variable}
                   & \textbf{Description}
                  \\ \midrule[.3mm]
                  $\delay \gets 0$
                   & The maximum network latency.
                  \\ \midrule
                  $\partyset \gets \emptyset$
                   & The set of registered parties.
                  \\ \midrule
                  $\vecM \gets [] $
                   & A dynamically updatable list of quadruples $(m, \msgid, D_{\msgid}, \party)$ where $D_{\msgid}$ denotes the fetch counter.
                  \\ \bottomrule[.3mm]
            \end{tabularx}
            \addtocounter{table}{-1}
      \end{minipage}

      \medskip
      \paragraph{Setting the delay:}
      %
      \begin{cccItemize}[nosep]
            \item Upon  receiving $(\textsc{set-delay}, \sid, d)$ from the adversary \adv, \textbf{if} \textsc{set-delay} has never been received \textbf{then} set $\delay = d$. Return $(\textsc{set-delay}, \sid, \ok)$ to \adv.
      \end{cccItemize}

      \paragraph{Network capabilities:}
      %
      \begin{cccItemize}[nosep]
            \item Upon receiving $(\textsc{diffuse}, \sid, m)$ from some
            $\party_s \in \partyset$, where $\partyset = \{ \party_1, \ldots, \party_n \}$ denotes the current party set, do:
            %
            \begin{cccEnum}[nosep]
                  \item Choose $n$ new unique message-IDs $\msgid_1, \ldots, \msgid_n$.
                  \item Initialize $2n$ new variables $D_{\msgid_1} := D^{MAX}_{\msgid_1} \ldots := D_{\msgid_n} := D^{MAX}_{\msgid_n} := 1$ and a per message-delay $\delay_{\msgid_i} = \delay$ for $i \in [n]$.
                  \item Set  $\vecM := \vecM \concat (m, \msgid_1, D_{\msgid_1}, \party_1) \concat \ldots \concat (m, \msgid_n, D_{\msgid_n}, \party_n)$.
                  \item Send $(\textsc{diffuse}, \sid, m, \party_s, (\party_1, \msgid_1), \ldots ,(\party_n, \msgid_n))$ to the adversary.
            \end{cccEnum}

            \item Upon receiving $(\textsc{fetch}, \sid)$ from $\party \in \partyset$, or from \adv on behalf of a corrupted party \party:
            %
            \begin{cccEnum}[nosep]
                  \item For all tuples $(m, \msgid, D_\msgid, \party) \in \vecM$, set $D_\msgid := D_\msgid - 1$.
                  \item Let $\vecM^\party_0$ denote the subvector \vecM including all tuples of the form $(m, \msgid, D_\msgid, \party)$ with $D_\msgid \le 0$ (in the same order as they appear in \vecM).
                  %
                  Delete all entries in $\vecM^\party_0$ from \vecM and in case some $(m, \msgid, D_\msgid, \party)$ is in
                  $\vecM^\party_0$, where \party is honest, set $\delay_{\msgid'} = \delay$ for any $(m, \msgid', D_{\msgid'}, (\cdot, \sid))$ in \vecM and replace this record by $(m, \msgid', \min \{ D_{\msgid'}, \delay \}, \party')$.
                  \item Output $\vecM^\party_0$ to \party (if \party is corrupted, send $\vecM^\party_0$ to \adv).
            \end{cccEnum}
      \end{cccItemize}

      \paragraph{Additional adversarial capabilities:}
      %
      \begin{cccItemize}[nosep]
            \item Upon receiving $(\textsc{diffuse}, \sid, m)$ from some corrupted $\party_s \in \partyset$ (or from \adv on behalf of $\party_s$ if corrupted), 
            % do 
            execute it the same way as an honest-sender diffuse, with the only difference that $\delay_{\msgid_i} = \infty$.

            \item Upon receiving $(\textsc{delays}, \sid, (T_{\msgid_{i_1}}, \msgid_{i_1}), \ldots, (T_{\msgid_{i_\ell}}, \msgid_{i_\ell}))$ from the adversary do the following for each pair $(T_{\msgid_{i_j}}, \msgid_{i_j})$:
            %
            if $D^{MAX}_{\msgid_{i_j}} + T_{\msgid_{i_j}} \le \delay_{\msgid_{i_j}}$ and $\msgid_{i_j}$ is a message-ID of receiver $\party = (\cdot, \sid)$ registered in the current \vecM, set $D_{\msgid_{i_j}} := D_{\msgid_{i_j}} + T_{\msgid_{i_j}}$ and set $D^{MAX}_{\msgid_{i_j}} := D^{MAX}_{\msgid_{i_j}} + T_{\msgid_{i_j}}$; otherwise, ignore this pair.

            \item Upon receiving $(\textsc{swap}, \sid, \msgid, \msgid')$ from the adversary, if \msgid and $\msgid'$ are message-IDs registered in the current \vecM, then swap the triples $(m, \msgid, D_\msgid, (\cdot, \sid))$ and $(m, \msgid', D_{\msgid'}, (\cdot, \sid))$ in \vecM.
            %
            Return $(\textsc{swap}, \sid)$ to the adversary.

            \item Upon receiving $(\textsc{get-reg}, \sid)$ from \adv, return the response $(\textsc{get-reg}, \sid, \partyset)$ to \adv.
      \end{cccItemize}
\end{cccFunctionality}
