\section{The \textsf{Timekeeper} Protocol}
\label{sec:complete-protocol-description}

In this section we give the full specification of the \textsf{Timekeeper} protocol.
%
The description is presented in UC-like notation.

\paragraph{The main protocol instance.}
%
We first introduce the main \timekeeper protocol instance that dispatches to the relevant subprocesses.

\begin{cccProtocol}
    {$\mathsf{Timekeeper}(\party, \sid; \funcLedger, \funcImpClock, \funcRO, \funcDiffuse)$}
    {protocol-instance}
    {The main protocol instance of \timekeeper}

    \noindent\textbf{Global Variables:}
    %
    \begin{cccItemize}[nosep]
        \item Read-only: \syncLen, \diffLen, $t_{\mathsf{off}}$, $t_{\mathsf{gather}}$, $t_{\mathsf{pre}}$

        \item Read-write: \localTime, \epoch, \round, \chainLocal, $T^\epoch_\party$, $\mathtt{isInit}$, $t_{\mathsf{work}}$, \buffer, \futureChains, \isSync, $\mathtt{fetchCompleted}$, $\mathtt{lastTimeAlert}$, $\mathsf{arrivalTime}_{SB}(\cdot)$.
    \end{cccItemize}

    \paragraph{Registration / Deregistration:}
    %
    \begin{cccItemize}[nosep]
        \item Upon receiving input $(\textsc{register}, \mathcal{R})$, where $\mathcal{R} \in \{ \funcLedger, \funcImpClock \}$ execute protocol $\textsf{Registration-Timekeeper}(\party, \sid, \mathtt{Reg}, \mathcal{R})$.

        \item Upon receiving input $(\textsc{de-register}, \mathcal{R})$, where $\mathcal{R} \in \{ \funcLedger, \funcImpClock \}$ execute protocol $\textsf{Deregistration-Timekeeper}(\party, \sid, \mathtt{Reg}, \mathcal{R})$.

        \item Upon receiving input $(\textsc{is-registered}, \sid)$ return $(\textsc{register}, \sid, 1)$ if the local registry \texttt{Reg} indicates that this party has successfully completed a registration with $\mathcal{R} = \funcLedger$ (and did not de-register since then).
        %
        Otherwise, return $(\textsc{register}, \sid, 0)$.
    \end{cccItemize}

    \paragraph{Interacting with the Ledger:}
    %
    Upon receiving a ledger-specific input $I \in \{(\textsc{submit}, \ldots),\allowbreak (\textsc{read}, \ldots),(\textsc{maintain-ledger}, \allowbreak \ldots)\}$ verify first that all resources are available.
    %
    \textbf{If} not all resources are available, \textbf{then} ignore the input; \textbf{else} (i.e., the party is operational and time-aware) execute one of the following steps depending on the input $I$:
    %
    \begin{cccItemize}[nosep]
        \item \textbf{If} $I = (\textsc{submit}, \sid, \tx)$ \textbf{then} set $\buffer \gets \buffer \concat \tx$, and send $(\textsc{diffuse}, \sid, \tx)$ to \funcDiffuse.

        \item \textbf{If} $I = (\textsc{maintain-ledger}, \sid, \mathrm{minerID})$ \textbf{then} invoke protocol $\mathsf{LedgerMaintenance} (\party, \sid)$; \textbf{if} \textsf{LedgerMaintenance} halts \textbf{then} halt the protocol execution (all future input is ignored).

        \item \textbf{If} $I = (\textsc{read}, \sid)$ \textbf{then} invoke protocol $\textsf{ReadState}(\party, \sid)$.

        \item \textbf{If} $I = (\textsc{export-time}, \sid)$ \textbf{then} do the following:
        %
        If \isSync or $\mathtt{isInit}$ is \false, then return $(\textsc{export-time}, \sid, \bot)$ to the caller.
        %
        Otherwise, call $\textsf{UpdateTime}(\party, R)$ and return $(\textsc{export-time}, \sid, \localTime)$ to the caller.
    \end{cccItemize}

    \paragraph{Handling calls to the shared setup:}
    %
    \begin{cccItemize}[nosep]
        \item Upon receiving $(\textsc{clock-tick}, \sid_C)$, forward it to \funcImpClock and output \funcImpClock's response.

        \item Upon receiving $(\textsc{clock-update}, \sid_C)$, record that a clock-update was received in the current round.
        %
        If the party is registered to all its setups, then do nothing further.
        %
        Otherwise, do the following operations \textit{before concluding this round}:
        %
        \begin{cccEnum}[nosep]
            \item If this instance is currently time-aware but otherwise stalled or offline, then call $\textsf{UpdateTime}(\party, R)$ to update \localTime.
            %
            If the party has passed a synchronization interval, then set $\isSync \gets \false$.

            \item If this instance is only stalled but $\isSync = \true$, then additionally execute $\textsf{FetchInformation}(\party, \sid)$, extract all new synchronization beacons from the fetched chains and record their arrival times and set $\mathtt{fetchCompleted} \gets \true$.
            %
            Also, any unfinished interruptible execution of this round is marked as completed.

            \item Forward $(\textsc{clock-update}, \sid_C)$ to \funcImpClock to finally conclude the round.
        \end{cccEnum}
    \end{cccItemize}
\end{cccProtocol}

\paragraph{Registration and de-registration.}
%
In order to perform basic operations, a party \party needs to register to all resources.
%
Note that \party is aware whether he is not synchronized not and will set the bit variable \isSync correspondingly.

\begin{cccProtocol}
    {$\mathsf{Registration\text{-}Timekeeper} (\party, \sid, \mathtt{reg}, \mathcal{G})$}
    {registration}
    {Parties register on necessary resources to run \timekeeper.}

    \begin{algorithmic}[1]
        \oneLineIf{$\mathcal{G} = \funcImpClock $}{send $(\textsc{register}, \sid)$ to $\mathcal{G}$, set registration status to registered with $\mathcal{G}$, and output the value received by $\mathcal{G}$.}

        \If{$\mathcal{G} = \funcLedger$}
        \If{the party is not registered with \funcImpClock or already registered with all setups}
        \State{ignore this input.}
        \Else
        \State{Send $(\textsc{clock-tick}, \sid_C )$ to \funcImpClock and receive $(\textsc{clock-tick}, \sid_C , \text{tick})$.}

        \State{Send $(\textsc{register}, \sid)$ to \funcDiffuse.}

        \State{Set $\localTime := \protocolTime{1}{1}$ and $\isSync \gets \false$.}

        \State{If this is the first registration invocation for this ITI, then set $\mathtt{isInit} \gets \false$.}

        \State{Output $(\textsc{register}, \sid, \party)$ once completing the registration with all above resources \F.}
        \EndIf
        \EndIf
    \end{algorithmic}
\end{cccProtocol}

The deregistration process is an analogous action.
%
Note that parties will record the last alert time which might be used to synchronize if it is only stalled.

\begin{cccProtocol}
    {$\mathsf{Deregistration\text{-}Timekeeper} (\party, \sid, \mathtt{reg}, \mathcal{G})$}
    {deregistration}
    {Parties de-register from corresponding resources in \timekeeper.}

    \begin{algorithmic}[1]
        \State{If the party is alert, set $\mathtt{lastTimeAlert} \gets\localTime$.}
        \If{$\mathcal{G} = \funcImpClock$}
        \State{Set $\isSync \gets \false$}
        \State{Send $(\textsc{de-register}, \sid)$ to $\mathcal{G}$ and set registration status as de-registered with $\mathcal{G}$}
        \State{Output the value received by $\mathcal{G}$}
        \EndIf

        \If{$\mathcal{G} = \funcLedger$}
        \State{Set $\isSync \gets \false$}
        \State{Send $(\textsc{de-register}, \sid)$ to \funcDiffuse, set its registration status as de-registered with \funcDiffuse and output $(\textsc{de-register}, \sid, \party)$.}
        \EndIf
    \end{algorithmic}
\end{cccProtocol}

\paragraph{Ledger Maintenance}
%
We group all the steps in the main ledger operation in \textsf{LedgerMaintenance}.
%
Note that what a party might execute depends on its status.

\begin{cccProtocol}
    {$\mathsf{LedgerMaintenance} (\party, \sid)$}
    {ledger-maintenance}
    {The main operations for parties to maintain the ledger.}

    \begin{algorithmic}[1]
        \LineComment{The following steps are executed in an $(\textsc{maintain-ledger}, \sid, \text{minerID})$-interruptible manner:}

        \oneLineIf{$\mathtt{isInit} = \false$}{Send $(\textsc{Retrieve}, \sid)$ to \funcCRS and receive $(\textsc{Retrieved}, \mathsf{CRS})$.}

        \If{\isSync and stalled before (and now up and running)}
        \State{Call $\mathsf{SimulateClockAdjustments}(\party, \sid)$}
        \EndIf

        \oneLineIf{\textbf{not} \isSync}{Call $\textsf{JoinProc}(\textsf{P}, \sid)$}

        \LineComment{Normal operation when alert}

        \State{Call $\textsf{FetchInformation}(\party, \sid)$ and denote the output by $(\chain_1, \ldots , \chain_N), (\tx_1, \ldots, \tx_k)$}

        \State{Set $\buffer \gets \buffer \concat (\tx_1, \ldots , \tx_k)$ and define $\futureChains \gets \futureChains \concat (\chain_1, \ldots , \chain_N)$}
        \State{Call $\textsf{UpdateTime}(\party, \sid)$}

        \LineComment{Ensures the processing of new beacons arrived in chains only.}

        \State{Extract beacons $B \gets \{\beacon_1, \ldots , \beacon_n\}$ contained in $\chain_1, \ldots , \chain_N$ and not yet contained in \syncBuffer.}

        \State{Call $\mathsf{ProcessBeacons}(\party, \sid, B)$}

        \State{Let $\mathcal{N}_0$ be the subsequence of \futureChains s.t. $\chain \in \mathcal{N}_0 :\Leftrightarrow \forall \block \in \chain : \timestamp{\block} \le \localTime$}

        \State{Remove each $\chain \in \mathcal{N}_0$ from \futureChains.}

        \State{$\mathtt{fetchCompleted} \gets \true$}

        \State{Call $\mathsf{SelectChain}(\party, \sid, \chainLocal, \mathcal{N}_0, \syncLen)$ to update \chainLocal}

        \If{$t_{\mathsf{work}} < \localTime$}

        \State{Call $\mathsf{UpdateMiningTarget}(\party, \diffLen)$ to update $T^\chain_\epoch$}

        \State{Call $\textsf{UpdateFreshRandomness}(\party, \syncLen)$ to update $\eta_\interval$}

        \State{Call $\mathsf{MiningProcedure}(\party, \epoch, \round, \buffer, \chainLocal)$}

        \State{Set $t_{\mathsf{work}} \gets \localTime$}

        \oneLineIf{$\round = \interval \cdot \syncLen$}{Call $\mathsf{SyncProcedure}(\party, \sid, R)$}
        
        \EndIf

        \State{Call $\textsf{FinishRound}(\party)$ \Comment{Mark normal round actions as finished.}}
    \end{algorithmic}
\end{cccProtocol}

\paragraph{Chain verification.}
%
A core procedure is to distinguish valid from invalid chains.
%
The procedure is depicted below.
%
It adopts the verification rule of the main blockchain structure similar to that in Bitcoin (cf.~\cite{C:GarKiaLeo17,EPRINT:GarKiaLeo20}), and extends it with beacon verifications.

Recall that \Isync{\interval} defined in~\cref{eq:isync-def} is a useful function which extracts the valid timestamp set in beacon mining and inclusion phase w.r.t. interval \interval.
%
In addition, \textsf{isvalidate} is a predicate that checks whether transactions in the blockchain achieves a valid ledger state (for details, see~\cite{C:BMTZ17}).

\begin{cccAlgorithm}
    {$\textsf{IsValidChain}(\chain)$}
    {is-valid-chain}
    {The chain validation procedure.}

    \begin{algorithmic}[1]
        \oneLineIf{\chain contains empty intervals or starts with a block with hash reference other than $\mathsf{CRS}$, or $\mathsf{isvalidstate}(\overrightarrow{\st}) = 0$}{\Return \false}

        \oneLineIf{\isSync and $(\exists \block \in \chain : \timestamp{\block} > \localTime)$}{\Return \false}

        \For{each interval $\interval'$}

        \State{$\epoch' \gets \mathsf{TargetRecalcEpoch}(\protocolTime{\interval'}{\cdot})$}

        \LineComment{Derive target and randomness for interval $\interval'$}

        \State{Set $T_{\epoch'}^\chain$ to be the target for epoch $\epoch'$ in \chain.}

        \State{Set $\eta_{\interval'} \gets G(\eta_{\interval' - 1} \concat \interval' \concat v)$ where $v$ is the concatenation of all block hash in interval $\interval' - 1$, and $\eta_1 \triangleq \mathsf{CRS}$.}

        \For{each block \block in \chain from interval $\interval'$} \Comment{check 2-for-1 PoW}
        \State{Parse \block as $\langle ctr, h, \st, \protocolTime{\interval'}{\round'}, txLabel \rangle$.}

        \LineComment{Check hash}
        \State{Set $\mathsf{badhash} \gets (h \neq H(\block^{-1}))$, where $\block^{-1}$ is the last block in \chain before \block.}

        \LineComment{Check nonce}
        \State{Set $\mathsf{badnonce} \gets ( H(ctr, h, \st, \protocolTime{\interval'}{\round'}, txLabel) < T_{\epoch'}^\chain) \wedge (ctr < 2^{32})$}

        \LineComment{Check beacons}
        \If{$\exists \beacon \in \block$ and $\timestamp{\beacon} \notin \Isync{\interval'}$}

        \State{Set $\mathsf{badBeacon} \gets \true$}

        \ElsIf{$\exists \beacon \in \block : \timestamp{\beacon} > \timestamp{\block}$}

        \State{Set $\mathsf{badBeacon} \gets \true$}

        \Else
        \State{Parse \beacon as $\langle \protocolTime{\interval'}{\round'}, \party', \eta_\beacon, ctr, blockLabel \rangle$}

        \oneLineIf{\chain contains more than one beacon with $(\round', \party', \cdot)$}{set $\mathsf{badBeacon} \gets \true$}

        \oneLineIf{$\eta_\beacon \neq \eta_{\interval'}$ \textbf{or} $\round' \notin \Isync{\interval'}$}{set $\mathsf{badBeacon} \gets \true$}

        \State{$u \gets H(ctr, \round', blockLabel, \protocolTime{\interval'}{\round'}, \party', \eta_{\interval'})$}

        \State{Set $\mathsf{badBeacon} \gets ( \stringRev{u} < T_{\epoch'}^\chain ) \wedge (ctr' < 2^{32})$}

        \EndIf

        \oneLineIf{$\sf (badhash \vee badnonce \vee badBeacon)$}{\Return \false}

        \EndFor
        \EndFor
        
        \State{\Return \true}
    \end{algorithmic}
\end{cccAlgorithm}

\paragraph{The beacon validity predicate.}
%
Beacons validity is related to chain validity as one need the corresponding target as well as fresh randomness to check beacon validity.
%
The details can be found below.

\begin{cccAlgorithm}
    {$\textsf{ValidSB}(\beacon, \chain)$}
    {valid-sb}
    {The beacon validation procedure.}

    \begin{algorithmic}[1]
        \LineComment{Precondition: Chain \chain is valid. Returns \true if the beacon is a valid beacon w.r.t. \chain, $\mathsf{undecided}$ if no judgement is possible, and \false if the beacon is invalid w.r.t. \chain.}

        \State{Parse \beacon as $\langle \protocolTime{\interval'}{\round'}, \party', \eta_\beacon, ctr, blockLabel \rangle$}

        \If{\chain contains no block in interval $\interval'$}
        \State{\Return $\mathsf{undecided}$}
        \Comment{no judgement possible for this beacon}
        \EndIf

        \LineComment{Derive target and randomness for interval $\interval'$ as indicated by \chain}

        \State{$\epoch' \gets \mathsf{TargetRecalcEpoch}(\protocolTime{\interval'}{\round'})$}

        \State{Set $T_{\epoch'}^\chain$ to be the target for epoch $\epoch'$ in \chain.}

        \State{Set $\eta_{\interval'} \gets G(\eta_{\interval' - 1} \concat \interval' \concat v)$ where $v$ is the concatenation of all block hash in interval $\interval' - 1$, and $\eta_1 \triangleq \mathsf{CRS}$.}

        \LineComment{Check nonce value and freshness}

        \State{$u \gets H(ctr, \round', blockLabel, \protocolTime{\interval'}{\round'}, \party', \eta_{\interval'})$}

        \oneLineIf{$( \stringRev{u} < T_\epoch^\chain) \wedge (ctr' < 2^{32}) \wedge (\eta_{\interval'} = \eta_\beacon)$}{\Return \true}

        \State{\Return \false}
    \end{algorithmic}
\end{cccAlgorithm}

\paragraph{Fetch information.}
%
Parties fetch information from two diffusion network --- $\funcDiffuse^{\mathsf{bc}}$ and $\funcDiffuse^{\mathsf{tx}}$ --- to learn new chains and transactions.

\begin{cccProtocol}{
    $\mathsf{FetchInformation}(\party, \sid)$}
    {fetch-information}
    {Fetching new blocks and transactions from $\funcDiffuse^{\mathsf{bc}}$ and $\funcDiffuse^{\mathsf{tx}}$.}
    
    \begin{algorithmic}[1]
        \oneLineIf{$\mathtt{fetchCompleted}$}{\Return}

        \LineComment{Fetch on $\funcDiffuse^{\mathsf{bc}}$}

        \State{Send $(\textsc{fetch}, \sid)$ to $\funcDiffuse^{\mathsf{bc}}$; denote the response from $\funcDiffuse^{\mathsf{bc}}$ by $(\textsc{fetch}, \sid, b)$.}

        \State{Extract chains $\chain_1, \ldots , \chain_k$ from $b$.}

        \LineComment{Fetch on $\funcDiffuse^{\mathsf{tx}}$}

        \State{Send $(\textsc{fetch}, \sid)$ to $\funcDiffuse^{\mathsf{tx}}$; denote the response from $\funcDiffuse^{\mathsf{tx}}$ by $(\textsc{fetch}, \sid, b)$.}

        \State{Extract received transactions $\tx_1, \ldots , \tx_k$ from $b$.}

        \If{\textbf{not} \isSync or \party is stalled}
        \State{$\buffer \gets \buffer \concat (\tx_1, \ldots, \tx_n)$}

        \State{$\futureChains \gets \futureChains \cup \{ \chain_1, \ldots, \chain_n \}$}
        \EndIf
    \end{algorithmic}

    \textsc{Output}: The protocol outputs $(\chain_1, \ldots , \chain_k)$ and $(\tx_1, \ldots , \tx_k)$ to its caller (but not to \Z).
\end{cccProtocol}

\paragraph{Update mining target, fresh randomness and local time.}
%
Following~\cref{eq:next-target} and the rule regarding dampening filter $\tau$, the targets for each epoch \epoch are computed as following.
%
Recall that \textsf{EpochBlockCount} defined in~\cref{eq:epochblockcount} outputs the number of blocks in the corresponding epoch, and $T_0$ is a pre-determined target value (hardcoded in the protocol) that guarantees a ``safe'' start in our assumption.

\begin{cccProtocol}
    {$\mathsf{UpdateMiningTarget}(\party, \sid)$}
    {update-mining-target}
    {Updating mining target based on local time and chain.}

    \begin{algorithmic}[1]
        \State{$\epoch \gets \mathsf{TargetRecalcEpoch}(\localTime)$}

        \If{$\epoch = 1$}
        
        \State{$T_\epoch \gets T_0$}
        
        \Else
        
        \State{$\varLambda = | \mathsf{EpochBlocks}(\chainLocal, \epoch - 1) |$}
        
        \State{$\varLambda = \min \{ \max\{\varLambda, \varLambda_{\mathsf{epoch}} / \tau \}, \varLambda_{\mathsf{epoch}} \cdot \tau \}$}
        
        \State{$T_\epoch \gets (\varLambda_{\mathsf{epoch}} / \varLambda) \cdot T_{\epoch - \diffLen / \syncLen}$}
        
        \EndIf
    \end{algorithmic}

    \textsc{Output}: The protocol outputs $T_\epoch$ to its caller (but not to \Z).
\end{cccProtocol}

Parties will extract the fresh randomness in every synchronization interval from the block hash in previous epoch.
%
Note that $G(\cdot)$ is another hash function different from $H(\cdot)$ therefore it does not interact with the random oracle \funcRO in the mining procedure.

\begin{cccProtocol}
    {$\textsf{UpdateFreshRandomness}(\party, \sid)$}
    {update-fresh-randomness}
    {Updating mining target based on local chain}

    \begin{algorithmic}[1]
        \State{Set $\eta_\interval \gets G(\eta_{\interval - 1} \concat \interval \concat v)$ where $v$ is the concatenation of all block hash in interval $\interval - 1$, and $\eta_1 \triangleq \mathsf{CRS}$.}
    \end{algorithmic}

    \textsc{Output}: The protocol outputs $\eta_\epoch$ to its caller (but not to \Z).
\end{cccProtocol}

Parties will send \textsc{clock-tick} to \funcImpClock to check if it receives a $\mathrm{tick} = 0$, which indicates the beginning of a new (local) round.
%
Note that alert parties will never change the interval index \interval here when adding $1$ to \localTime (regarding the rules of addition, see~\cref{subsec:protocol-timestamps}); they will only adjust \interval in \textsf{SyncProcedure} (code in~\cref{protocol:sync-proc}).
%
Meanwhile, for those stalled parties, their local time will increase as if no adjustment happens.

\begin{cccProtocol}
    {$\textsf{UpdateTime}(\party, \syncLen)$}
    {update-time}
    {Updating local time from \funcImpClock.}

    \begin{algorithmic}[1]
        \LineComment{Precondition: Only executed if time-aware.}

        \State{Send $(\textsc{clock-tick}, \sid_C)$ to \funcImpClock and receive $(\textsc{clock-tick}, \sid_C , \mathrm{tick})$}

        \If{$\mathrm{tick} = 0$}

        \State{$\localTime \gets \localTime + 1$}

        \State{$\mathtt{fetchCompleted} \gets \false$}

        \EndIf

        \State{$\epoch \gets \mathsf{TargetRecalcEpoch}(\localTime)$}
    \end{algorithmic}

    \textsc{Output}: The protocol outputs \localTime, \epoch to its caller (but not to \Z).
\end{cccProtocol}

\paragraph{Process beacons and arrival times.}
%
The following procedure processes incoming beacons, bookkeeps their arrival times and filters out invalid as well as duplicate beacons.
%
The predicate to verify beacons is presented in~\cref{algorithm:valid-sb}.
%
Regarding the duplicate beacons recording the same timestamp and miner identity, only one with the earliest arrival time will be preserved.

\begin{cccProtocol}
    {$\mathsf{ProcessBeacons}(\party, \sid, B)$}
    {process-beacons}
    {Parties filter invalid synchronization beacons.}

    \begin{algorithmic}[1]
        \oneLineIf{$\mathtt{fetchCompleted}$}{\Return}
        \State{Send $(\textsc{fetch}, \sid)$ to $\funcDiffuse^{\mathsf{sync}}$. denote the response from $\funcDiffuse^{\mathsf{sync}}$ by $(\textsc{fetch}, \sid, b)$}

        \State{Extract all received beacons $(\beacon_1, \ldots , \beacon_k)$ contained in $b \cup B$.}

        \For{each $\beacon_i$ with $\mathsf{arrivalTime}_{SB}(\beacon_i) = \bot$}
        \State{$\syncBuffer \gets \syncBuffer \cup \{ \beacon_i \}$}

        \State{Let $\interval'$ be the interval index $\timestamp{\beacon_i}$ belongs to}

        \If{$\isSync \wedge (\interval \ge \interval')$}

        \State{Set $\mathsf{arrivalTime}_{SB}(\beacon_i) \gets (\localTime, \mathsf{final})$}
        \Comment{The measurement is final.}

        \Else \Comment{Will be adjusted upon next time shift.}
        \State{$\mathsf{arrivalTime}_{SB}(\beacon_i) \gets (\localTime, \mathsf{temp})$}

        \EndIf
        \EndFor

        \LineComment{Buffer cleaning. Keep one representative arrival time.}

        \If{\isSync}
        \State{Remove from \syncBuffer all beacons s.t. $\textsf{ValidSB}(\party, \sid, \beacon, \chainLocal, \syncLen)$ returns \false}

        \State{$\syncBuffer_{\mathrm{valid}} \gets \{\beacon' \in \syncBuffer \mathbin{|} \textsf{ValidSB}(\party, \sid, \beacon', \chainLocal, \syncLen) = \true \}$}

        \State{Let $L = (\beacon_1, \ldots , \beacon_n)$ be a canonical ordering of $\syncBuffer_{\mathrm{valid}}$}

        \For{each $\beacon = \langle \protocolTime{\interval'}{\round'}, \party', \eta_\beacon, ctr, blockLabel \rangle \in L$}
        \State{$Q_\beacon \gets \{ \beacon' = \langle \protocolTime{\interval''}{\round''}, \party'', \cdot, \cdot, \cdot \rangle \in L \mathbin| \party' = \party'' \wedge  \protocolTime{\interval'}{\round'} =  \protocolTime{\interval''}{\round''} \}$}

        \State{$\min_\beacon \gets \min \{ \mathsf{arrivalTime}(\beacon') \mathbin| \beacon' \in Q_\beacon \}$}

        \State{$\beacon' \gets \min \{\beacon'' \in Q_\beacon \mathbin| \mathsf{arrivalTime}(\beacon'') = \min_\beacon \}$}
        \Comment{Min w.r.t. ordering in $L$}

        \State{Remove from \syncBuffer all beacons $\langle \protocolTime{\interval'}{\round'}, \party', \cdot, \cdot, \cdot \rangle$ except $\beacon'$}
        \EndFor
        \EndIf
    \end{algorithmic}

    \textsc{Output}: \textsf{ok} to its caller (but not to \Z).
\end{cccProtocol}

\paragraph{Chain selection.}
%
Parties drop invalid chains, and then select the a chain by \emph{heaviest difficulty} chain selection rule (cf.~\textsf{maxvalid} in~\cite{C:GarKiaLeo17}).
%
More specifically, $\max(\chain_1, \chain_2)$ will return the \emph{most difficult} of the two.
%
In case $\chainDiff{\chain_1} = \chainDiff{\chain_2}$, $\max(\cdot, \cdot)$  always return the first operand to reflect the fact that parties adopt the first chain they obtain from the network.

\begin{cccProtocol}
    {$\textsf{SelectChain}(\party, \sid, \mathcal{N} = \{ \chain_1, \ldots , \chain_n \})$}
    {select-chain}
    {Parties filter invalid chains and apply the heaviest chain selection rule.}

    \begin{algorithmic}[1]
        \State{Set $\chain_{\max} \gets \chainLocal$}

        \For{$i = 1$ \textbf{to} $n$}
        \oneLineIf{$\textsf{IsValidChain}(\chain_i)$ returns \true}{$\chain_{\max} \gets \max(\chain_{\max}, \chain_i)$}
        \EndFor

        \State{Replace \chainLocal by $\chain_{\max}$.}
    \end{algorithmic}

    \textsc{Output}: The protocol outputs $\chain_{\max}$ to its caller (but not to \Z).
\end{cccProtocol}

\paragraph{Mining Procedure}
%
Once a party \party has prepared all information and updated its state, it can run the core mining procedure formally given below.
%
When \localTime reports a timestamp that satisfies $\round \in \Isync{\interval}$ (i.e., \party stays in the beacon mining and inclusion phase), \party will include the fresh beacons and check if he succeeds in the beacon mining procedure.

Note that, in the ``prepare block content'' part, we follow~\cite{C:BMTZ17} and adopt several predicates such as $\mathsf{blockify}_{\mathsf{OC}}$ and $\mathsf{validTX}_{\mathsf{OC}}$ to generate the Merkle root (\st) of the block content.
%
$\mathsf{blockify}_{\mathsf{OC}}$ is used to compute the root value, and $\mathsf{validTX}_{\mathsf{OC}}$ can justify whether an incoming transaction is valid w.r.t. the current blockchain state.
%
This part is not relevant to the topic of this paper; for further discussion, we refer to~\cite{C:BMTZ17}.

\begin{cccProtocol}
    {$\textsf{MiningProcedure}(\party, \sid)$}
    {mining-procedure}
    {The mining procedure of \timekeeper.}

    \begin{algorithmic}[1]
        \LineComment{The following steps are executed in an (\textsc{maintain-ledger}, \sid, minerID)-interruptible manner:}

        \oneLineIf{$h \neq H(\textsf{head}(\chainLocal))$}{$h \gets H(\textsf{head}(\chainLocal))$}
        \Comment{Check if switch to a new chain.}

        \LineComment{Prepare block content.}

        \State{Set $\buffer' \gets \buffer, \mathbf{N} \gets \tx^{\mathrm{base-tx}}_\party$ , and $\st \gets \textsf{blockify}_{\textsf{OC}}(\mathbf{N})$.}

        \Repeat
        \State{Parse $\buffer'$ as sequence $(\tx_1, \ldots, \tx_n)$}

        \For{$i = 1$ to $n$}
        \If{$\textsf{validTX}_{\textsf{OC}}(\tx_i, \overrightarrow{\st} \concat \st) = 1$}
        \State{$\textbf{N} \gets \textbf{N} \concat \tx_i$}

        \State{Remove $\tx'$ from $\buffer'$}

        \State{Set $\st \gets \textsf{blockify}_{\textsf{OC}}(\textbf{N})$}
        \EndIf
        \EndFor
        \Until{\textbf{N} does not increase any more}

        \LineComment{Check if a beacon should be included.}

        \If{$\Isync{\localTime} = \true$}
        \State{$B \gets \{\beacon' \in \syncBuffer \mathbin| \textsf{ValidSB}(\party, \sid, \beacon', \chainLocal, \syncLen) = \true \}$}

        \State{Remove from $B$ all beacons $\beacon = \langle \protocolTime{\interval'}{\round'}, \party', \cdot, \cdot, \cdot \rangle$ that satisfy $(\timestamp{\beacon} > \localTime) \vee (\Isync{\timestamp{\beacon}} = \true) \vee \chainLocal$ contains a beacon $\langle \protocolTime{\interval'}{\round'}, \party', \cdot, \cdot, \cdot \rangle$.}

        \State{$\textbf{N} \gets \textbf{N} \concat B$}

        \State{Set $\st \gets \textsf{blockify}_{\textsf{OC}}(\textbf{N})$}
        \EndIf

        \LineComment{prepare 2-for-1 PoW.}

        \State{$blockLabel \gets  h \concat \st, txLabel \gets \party \concat \eta_\interval $}

        \State{$u \gets H(ctr, h, \st, \localTime, \party, \eta_\interval)$}

        \LineComment{Check if block mining succeed.}

        \If{$(u < T^\chain_\epoch) \wedge (ctr < 2^{32})$}
        \State{Set $\block \gets \langle h, \st, \localTime, ctr, txLabel \rangle$ and update $\chainLocal \gets \chainLocal \concat \block$}

        \State{Send $(\textsc{diffuse}, \sid, \chainLocal)$ to $\funcDiffuse^{\mathsf{bc}}$}
        
        \State{Set anchor and proceed from here upon next activation of this procedure.}
        \EndIf

        \LineComment{Check if a PoW timestamp transaction mining succeed.}

        \If{$(\stringRev{u} < T^\chain_\epoch) \wedge (ctr < 2^{32}) \wedge (\Isync{\localTime} = \true)$}
        \State{$\beacon \gets \langle \localTime, \party, ctr, \eta_\interval, blockLabel \rangle$}

        \State{Send (\textsc{diffuse}, \sid, \beacon) to $\funcDiffuse^{\mathsf{sync}}$}
        \State{Set anchor at the end of the procedure to resume on next maintenance activation.}
        \EndIf

        \State{$ctr \gets ctr + 1$}
    \end{algorithmic}
\end{cccProtocol}

\paragraph{Synchronization Procedure}
%
The synchronization procedure is called when party's local clock enters a clock synchronization interval boundary (i.e., $\localTime = \protocolTime{\interval}{\interval \cdot \syncLen}$).
%
Note that thanks to the timestamp scheme in \timekeeper, parties will only pass the interval boundary for once.

\begin{cccProtocol}
    {$\mathsf{SyncProcedure}(\party, \sid)$}
    {sync-proc}
    {Parties update their local clocks.}

    \begin{algorithmic}[1]
        \LineComment{Only called when: \party is alert, $\localTime = \protocolTime{\interval}{\interval \cdot \syncLen}$ and $\interval > 0$}

        \State{$B \gets \{ \block \mathbin| (\block \in \chainLocal)  \wedge (\timestamp{\block} = \protocolTime{\interval}{\cdot}) \wedge (\Isync{\timestamp{\block}} = \true) \}$}

        \State{$SB \gets \{\beacon \mathbin| (\beacon \in \block \in B) \wedge (\timestamp{\beacon} = \protocolTime{\interval}{\cdot}) \wedge (\Isync{\timestamp{\beacon}} = \true) \}$}

        \LineComment{Find representative beacon and compute recommendation.}

        \For{each $\beacon =  \langle \protocolTime{\interval'}{\round'}, \party', ctr, \eta_\beacon, blockLabel \rangle \in SB$}

        \State{Find unique $\beacon' = \langle \protocolTime{\interval'}{\round'}, \party', \cdot, \cdot, \cdot \rangle \in \syncBuffer$. If inexistent, set $\beacon' \gets \bot$.}

        \If{$\beacon' \neq \bot$}
        \State{Set $\mathsf{arrivalTime}_{SB}(\beacon) \gets \mathsf{arrivalTime}_{SB}(\beacon')$}

        \State{$\textsf{recom}(\beacon) \gets \timestamp{\beacon} - \mathsf{arrivalTime}(\beacon)$}
        \Else
        \State{$\mathcal{S} \gets \mathcal{S} \mathbin \backslash \{ \beacon \}$}
        \Comment{Negligible probability event in execution.}
        \EndIf
        \EndFor

        \State{$\shift_i \gets \med \{ \textsf{recom}(\beacon) \mathbin{|} \beacon \in SB \}$}

        \For{each \beacon with $\mathsf{arrivalTime}_{SB}(\beacon) = (a, \mathsf{temp})$}
        \State{$\mathsf{arrivalTime}_{SB}(\beacon) \gets (a + \shift_i, \mathsf{final})$}
        \EndFor

        \oneLineIf{$\shift_i = 0$}{$\interval \gets \interval + 1$}
        \Comment{No adjustment, simply enter next epoch}

        \If{$\shift_i > 0$} \Comment{Move fast forward}
        \State{$\textsf{newTime} \gets \protocolTime{\interval + 1}{\round + \shift_i}, \localTime \gets \protocolTime{\interval + 1}{\round}, \mathcal{M}_{\mathrm{chains}} \gets \emptyset$}

        \While{$\localTime < \textsf{newTime}$}
        \State{$\localTime \gets \localTime + 1$} \Comment{increment round counter in localtime by 1}

        \State{Let $\mathcal{N}_0$ be the subsequence of \futureChains s.t. $\chain \in N_0 : \Leftrightarrow \forall \block \in \chain : \timestamp{\block} < \localTime$}

        \State{Remove each $\chain \in \mathcal{N}_0$ from \futureChains}

        \State{Call $\textsf{SelectChain}(\party, \sid, \chainLocal, \mathcal{N}_0, \syncLen)$ to update \chainLocal}
        \EndWhile

        \State{Send $(\textsc{diffuse}, \sid, \mathcal{M}_{\mathrm{chains}})$ to $\funcDiffuse^{\mathsf{bc}}$ and proceed from here upon next activation of this procedure.}
        \EndIf

        \oneLineIf{$\shift_i < 0$}{Set $\localTime \gets (\interval + 1, \round + \shift_i)$}
        \Comment{Set clock back}
    \end{algorithmic}

    \textsc{Output}: The protocol outputs \textsf{ok} to its caller (but not to \Z).
\end{cccProtocol}

\paragraph{Reading the Ledger State}
%
In order to read the ledger state, the party \party first processes all relevant information and then extracts the state (the settled ledger).

\begin{cccProtocol}
    {$\mathsf{ReadState}(\party, \sid)$}
    {read-state}
    {Parties read the ledger state.}

    \begin{algorithmic}[1]
        \If{$\mathtt{isInit} = \false \vee \isSync = \false$}
        \State{Output the empty state $(\textsc{read}, \sid, \varepsilon)$ (to Z).}
        \Else
        \State{Call $\textsf{FetchInformation}(\party, \sid)$ and denote the output by $(\chain_1, \ldots , \chain_N), (\tx_1, \ldots, \tx_k)$}

        \State{Set $\buffer \gets \buffer \mathbin \Vert (\tx_1, \ldots, \tx_k)$ and define $\mathcal{N} \gets \{ \chain_1, \ldots , \chain_N \}$}

        \State{Call $\textsf{UpdateTime}(\party, \syncLen)$}

        \State{Call $\textsf{ProcessBeacons}(\party, \sid)$}

        \State{Let $\mathcal{N}_0 := \{ \chain \in \mathcal{N} \cup \futureChains \mathbin| \forall \block \in \chain : \timestamp{\block} \le \localTime \}$}

        \State{Let $\mathcal{N}_1 := \{ \chain \in \mathcal{N} \mathbin| \exists \block \in \chain : \timestamp{\block} > \localTime \}$}

        \State{$\futureChains \gets (\futureChains \mathbin \backslash \mathcal{N}_0) \cup \mathcal{N}_1 $}

        \State{$\mathtt{fetchCompleted} \gets \true$}

        \State{Call $\textsf{SelectChain}(\party, \sid, \chainLocal, \mathcal{N}_0, \syncLen)$ to update \chainLocal}

        \State{Extract the state \st from the current local chain \chainLocal}

        \State{Output $(\textsc{read}, \sid, \st^{\lceil k})$ (to \Z)} \Comment{$\st^{\lceil k}$ denotes the prefix of \st by pruning blocks with timestamps reporting the last $k$ rounds}
        \EndIf
    \end{algorithmic}
\end{cccProtocol}

\paragraph{Simulate Clock Adjustments}
%
If parties are merely de-registered from the random oracle \funcRO (namely, stalled for a limited time), they can bootstrap easily to the reliable state and time.
%
Note that if parties are stalled, their \localTime is still updated in \textsf{UpdateTime} and by computing the distance of round numbers in $\mathtt{lastTimeAlert}$ and \localTime we get the exact number of rounds that have elapsed during the stall period.

\begin{cccProtocol}
    {$\mathsf{SimulateClockAdjustment}(\party, \sid)$}
    {simulate-clock-adjustment}
    {Stalled parties simulate clock adjustments.}

    \newcommand{\simulatedTime}{\ensuremath{\mathsf{simulatedTime}}\xspace}
    \begin{algorithmic}[1]
        \State{$\simulatedTime \gets \mathtt{lastTimeAlert}$}

        \State{Set $\round'$ as round index in \simulatedTime}

        \For{$\round - \round'$ iterations}
        \State{Let $\mathcal{N}_0$ be the subsequence of \futureChains s.t. $\chain \in \mathcal{N}_0 : \Leftrightarrow \forall \block \in \chain : \timestamp{\block} \le \simulatedTime$}

        \State{Remove each $\chain \in \mathcal{N}_0 $ from \futureChains.}

        \State{Emulate $\mathsf{SelectChain}(\party, \sid, \chainLocal, \mathcal{N}_0, \syncLen)$ with simulated time \simulatedTime (instead of \localTime) to update \chainLocal}

        \If{$\simulatedTime = \protocolTime{\interval}{\interval \cdot \syncLen}$ for interval \interval}
        \State{Emulate $\mathsf{SyncProcedure}(\party, \sid, \syncLen)$ on simulated time \simulatedTime (instead of \localTime)}
        \EndIf
        \EndFor

        \State{Set $\localTime \gets \simulatedTime$}

        \State{Set $t_{\mathsf{work}} \gets \localTime - 1$}
    \end{algorithmic}

    \textsc{Output}: The protocol outputs \textsf{ok} to its caller (but not to \Z).
\end{cccProtocol}

\paragraph{Round finish procedure.}
%
Once a party \party has done its actions in a round, \party claims finishing current round by calling \textsf{FinishRound} and sending \textsc{clock-update} to \funcImpClock.
%
For details of the way to interact with \funcImpClock in a UC treatment, see~\cite{EC:BGKRZ21}.

\begin{cccProtocol}
    {$\mathsf{FinishRound}(\party, \sid)$}
    {finish-round}
    {Finishing a round.}

    \begin{algorithmic}[1]
        \While{A $(\textsc{clock-update}, \Z)$ has not been received during the current round }
        \State{Give up activation (set the anchor here)}
        \EndWhile

        \State{Send $(\textsc{clock-update}, \sid_C)$ to \funcImpClock.}
        \Comment{Party will lose its activation here.}
    \end{algorithmic}
\end{cccProtocol}

\paragraph{The Joining Procedure}
%
Another main procedure of \timekeeper is \textsf{JoinProc}, where newly joint parties synchronize with other alert parties by passively listening to the protocol execution, building blockchain and processing beacons to derive a local time that is close to all alert parties.
%
The default value of the parameters in each phase are summarized in~\cref{table:join-proc-param}.

\begin{cccProtocol}
    {$\mathsf{JoinProc}(\party, \sid)$}
    {join-procedure}
    {The procedure for a fresh new party to join \timekeeper.}

    \begin{algorithmic}[1]
        \LineComment{Phase A, state-reset}

        \State{Call $\textsf{UpdateTime}(\party, \syncLen)$} \Comment{Align with newest round.}

        \If{$\localTime > \protocolTime{1}{1}$}
        \State{Set $\localTime \gets \protocolTime{1}{1}$}

        \State{$\mathtt{fetchCompleted} \gets \false, \futureChains \gets \emptyset, \buffer \gets \emptyset$}

        \State{Set beacon arrival timetable as empty array}
        \EndIf

        \LineComment{Phase B, chain-convergence}

        \While{$\localTime < \protocolTime{1}{1} + t_{\mathsf{off}}$}
        \If{$\mathtt{fetchCompleted} = \false$}
        \State{Call $\textsf{FetchInformation}(\party, \sid)$ and denote fetched chains by $\mathcal{N} := (\chain_1, \ldots , \chain_N)$}

        \State{Call $\mathsf{SelectChain}(\party, \sid, \chainLocal, \mathcal{N}, \syncLen)$ to update \chainLocal}

        \State{$\mathtt{fetchCompleted} \gets \true$}

        \State{$\textsf{FinishRound}(\party)$}
        \EndIf

        \State{Call $\textsf{UpdateTime}(P, R)$ to update \localTime}

        \EndWhile

        \LineComment{Phase C, beacon-gathering}

        \While{$\localTime \le \protocolTime{1}{1} + t_{\mathsf{off}} + t_{\mathsf{gather}}$}
        \If{$\mathtt{fetchCompleted} = \false$}
        \State{Call $\textsf{FetchInformation}(\party, \sid)$ and denote the output by $(\chain_1, \ldots , \chain_N), (\tx_1, \ldots, \tx_k)$}

        \State{Set $\buffer \gets \buffer \concat (\tx_1, \ldots , \tx_k)$}

        \State{Set $\futureChains \gets \futureChains \concat (\chain_1, \ldots , \chain_N)$}

        \State{Call \textsf{ProcessBeacons} to collect new beacons in this round.} \Comment{All arrival times are temporary}

        \State{Call $\textsf{SelectChain}(\party, \sid, \chainLocal, \futureChains, R)$ to update \chainLocal}

        \State{$\mathtt{fetchCompleted} \gets \true$}

        \State{$\textsf{FinishRound}(\party)$}
        \EndIf

        \State{Call $\textsf{UpdateTime}(P, R)$ to update \localTime}
        \EndWhile

        \LineComment{Phase D, shift-computation}

        \State{$\syncBuffer_{\mathrm{valid}} \gets \{\beacon' \in \syncBuffer \mathbin| \textsf{ValidSB}(\party, \sid, \beacon', \chainLocal, \syncLen) = \true \}$ }

        \State{Initialize $i := 0$.}

        \State{Set $i$ to be the minimum positive integer such that $\forall \beacon \in \block \in B  : \beacon \in \syncBuffer_{\mathrm{valid}} \wedge \mathsf{arrivalTime}(\beacon) \ge \protocolTime{1}{1} + t_{\mathsf{off}} + t_{\mathsf{pre}}$ where $B \gets \{ \block \mathbin| (\block \in \chainLocal)  \wedge (\timestamp{\block} = \protocolTime{i}{\cdot}) \wedge (\Isync{\timestamp{\block}} = \true) \}$. \label{code:join-proc-i}}
        \Comment{if no interval exists, $i$ is unchanged.}

        \If{$i \ge 1$}
        \For{at most $(t_{\mathsf{gather}} ~\mathrm{div}~\syncLen)$ iterations}

        \State{$B \gets \{ \block \mathbin| (\block \in \chainLocal)  \wedge (\timestamp{\block} = \protocolTime{i}{\cdot}) \wedge (\Isync{\timestamp{\block}} = \true) \}$}

        \State{$SB \gets \{\beacon \mathbin| \beacon \in \block \in B \wedge (\timestamp{\beacon} = \protocolTime{i}{\cdot}) \wedge (\Isync{\timestamp{\beacon}} = \true) \}$}

        \For{each $\beacon = \langle \protocolTime{\interval'}{\round'}, \party', ctr, \eta_\beacon, blockLabel \rangle \in SB$}

        \State{$Q_\beacon \gets \{\beacon' = \langle \protocolTime{\interval''}{\round''}, \party'', \cdot, \cdot, \cdot \rangle \in \syncBuffer_{\mathrm{valid}} \mathbin| \party' = \party'' \wedge \protocolTime{\interval'}{\round'} = \protocolTime{\interval''}{\round''} \}$}

        \If{$Q_\beacon \neq \emptyset$}
        \State{$\min_\beacon \gets \min \{ \mathsf{arrivalTime}(\beacon) \mathbin| \beacon \in Q_\beacon\}$}

        \State{$\mathsf{arrivalTime}_{SB}(\beacon) \gets (\min_\beacon, \mathsf{final})$}

        \State{$\textsf{recom}(\beacon) \gets \timestamp{\beacon} - \mathsf{arrivalTime}(\beacon)$}
        \Else
        \State{$\mathcal{S} \gets \mathcal{S} \backslash \{ \beacon \}$} \Comment{Negligible probability event in execution}
        \EndIf
        \EndFor

        \State{$\shift_i \gets \med \{ \textsf{recom}(\beacon) \mathbin| \beacon \in SB \}$}

        \For{each \beacon with $\mathsf{arrivalTime}_{SB}(\beacon) = (a, \mathsf{temp})$}
        \State{$\mathsf{arrivalTime}_{SB}(\beacon) \gets (a + \shift_i, \mathsf{temp})$}
        \EndFor

        \If{$\round + \shift_i \le (i + 1) \syncLen$}
        \State{Set $\localTime \gets \protocolTime{i + 1}{\round + \shift_i}$}
        \State{\textbf{Break}}
        \Else
        \State{Set $\localTime \gets \protocolTime{i + 1}{\round + \shift_i}$}
        \Comment{Temporarily invalid; will be adjust later.}
        \State{set $i \gets i + 1$} \Comment{continue iteration.}
        \EndIf

        \EndFor

        \State{$\isSync \gets \true$; run $\textsf{SelectChain}(\party, \sid, \chainLocal, \futureChains, \syncLen)$ to update \chainLocal; $t_{\mathsf{work}} \gets \protocolTime{\interval}{\round - 1}$}

        \For{each beacon $\beacon \in \syncBuffer_{\mathrm{valid}}$ with $\timestamp{\beacon} = \protocolTime{\interval}{\cdot}$}
        \State{Parse $\mathsf{arrivalTime}_{SB}(\beacon)$ as $(a, \mathsf{temp})$. Define $\mathsf{arrivalTime}_{SB}(\beacon) \gets (a, \mathsf{final})$}
        \EndFor

        \EndIf
    \end{algorithmic}
\end{cccProtocol}
