\section{The Clock Synchronization Protocol with Public Setup}
\label{sec:protocol-details}

In this section we present the general approach and the various core building blocks of the new clock synchronization protocol---\timekeeper.
%
For a complete description, refer to~\cref{sec:complete-protocol-description}.
%
At a high level, \timekeeper is a Nakamoto-style PoW-based blockchain protocol together with time synchronization functionalities.
%
Readers can think of it as a Bitcoin protocol with the following modifications (See~\cref{fig:protocol-execution} for an illustration of the protocol execution).

\begin{figure}[ht]
    \tikzstyle{intervalBox} = [draw = kanade, rectangle, fill = kanade!60, minimum width = 6em, minimum height = 2em, text = white, inner sep = .6em]

    \tikzstyle{epochBox} = [draw = kanade, rectangle, fill = kanade!80, minimum width = 24.15em, minimum height = 2em, text = white, inner sep = .6em]

    \begin{tikzpicture}
        \footnotesize
        \node[intervalBox] (interval1) {Interval 1};
        \foreach \x [count=\xi from 1] in {2,3,...,8}
            {
                \node[intervalBox, anchor = west] at (interval\xi.east) (interval\x) {Interval \x};
            }
        \node[intervalBox, minimum width = 1em, anchor = west] at (interval8.east) (interval9) {};

        \foreach \x in {1,2,...,8}
            {
                \node[circle, draw = kanade, fill = white, radius = .5em] at ([yshift = .7em]interval\x.north east) (clock\x) {};
                \draw[kanade] (clock\x.center) -- ([yshift = .4em]clock\x.center);
                \draw[kanade] (clock\x.center) -- ([xshift = .25em]clock\x.center);
            }

        \foreach \x [count=\xi from 1] in {1,5}
            {
                \node[epochBox, anchor = north west] at ([yshift = -3em]interval\x.south west) (epoch\xi) {Epoch \xi};
                \draw[kanade, line width = 1pt] ([yshift = -.2em]epoch\xi.south east) -- ([yshift = -1.2em]epoch\xi.south east);
                \node[draw=white, line width = .5pt, circle, fill = kanade, inner sep = .75pt] at ([yshift = -.9em]epoch\xi.south east) (rp11) {};
                \draw[kanade, line width = 1pt] ([xshift = -.4em, yshift = -.2em]epoch\xi.south east) -- ([xshift = -.4em, yshift = -1.2em]epoch\xi.south east);
                \node[draw=white, line width = .5pt, circle, fill = kanade, inner sep = .75pt] at ([xshift = -.4em, yshift = -.5em]epoch\xi.south east) (rp12) {};
                \draw[kanade, line width = 1pt] ([xshift = .4em, yshift = -.2em]epoch\xi.south east) -- ([xshift = .4em, yshift = -1.2em]epoch\xi.south east);
                \node[draw=white, line width = .5pt, circle, fill = kanade, inner sep = .75pt] at ([xshift = .4em, yshift = -.7em]epoch\xi.south east) (rp13) {};
            }
        \node[epochBox, minimum width = 1em, minimum height = 2.05em, anchor = north west] at ([yshift = -3em]interval9.south west) (epoch3) {};

        \draw[kanade] ([yshift = -1.5em]interval1.south west) -- ([xshift = 1.2em, yshift = -1.5em]interval8.south east);
        \node[fill = kanade, minimum size = .66em, inner sep = 0] at ([yshift = -1.5em]interval1.south west) (block1) {};
        \foreach \x [count=\xi from 1] in {.3, .5, .7, .4, .2, .8, .2, .1, .2, .5, .7, .3, .4, .5, .2, 1, 1.3, .3, .3, 1.3, .2, .9, .05, .05, .4, .3, .2, .2, .2, .15, .5, .5, .2, .2, .2, .3, .7, .3, 1.1, .7, .3, .2, .3}
            {
                \pgfmathtruncatemacro\xii{\xi + 1}
                \node[fill = kanade, minimum size = .66em, inner sep = 0, anchor = west] at ([xshift = \x em]block\xi.east) (block\xii) {};
            }

        \draw[line width = .5pt, kanade, densely dotted] (interval1.south west) -- ([yshift = -3em]interval1.south west);
        \foreach \x in {1,2,...,8}
            {
                \draw[line width = .5pt, kanade, densely dotted] (interval\x.south east) -- ([yshift = -3em]interval\x.south east);
            }

        \node[fill = white, rectangle, minimum width = 1em, minimum height = 7.5em, anchor = north west] at ([xshift = 1em, yshift = .2em]interval8.north east) (shade) {};

        \node[fill = kanade, minimum size = .66em, inner sep = 0, anchor = north] at ([yshift = -3em]epoch1.south east) (blockex) {};
        \node[text = kanade, anchor = west, inner sep = 0] at ([xshift = .2em]blockex.east) (blockextext) {\tiny Block};
        \node[circle, draw = kanade, fill = white, radius = .5em, anchor = west] at ([xshift = .5em]blockextext.east) (clockex) {};
        \draw[kanade] (clockex.center) -- ([yshift = .4em]clockex.center);
        \draw[kanade] (clockex.center) -- ([xshift = .25em]clockex.center);
        \node[text = kanade, anchor = west, inner sep = 0] at ([xshift = .2em]clockex.east) (clockextext) {\tiny Local Clock Adjustment};
        \draw[kanade, line width = 1pt] ([xshift = 1em, yshift = .5em]clockextext.east) -- ([xshift = 1em, yshift = -.5em]clockextext.east);
        \node[draw=white, line width = .5pt, circle, fill = kanade, inner sep = .75pt] at ([xshift = 1em, yshift = -.2em]clockextext.east) (rpex1) {};
        \draw[kanade, line width = 1pt] ([xshift = .6em, yshift = .5em]clockextext.east) -- ([xshift = .6em, yshift = -.5em]clockextext.east);
        \node[draw=white, line width = .5pt, circle, fill = kanade, inner sep = .75pt] at ([xshift = .6em, yshift = .2em]clockextext.east) (rpex2) {};
        \draw[kanade, line width = 1pt] ([xshift = 1.4em, yshift = .5em]clockextext.east) -- ([xshift = 1.4em, yshift = -.5em]clockextext.east);
        \node[draw=white, line width = .5pt, circle, fill = kanade, inner sep = .75pt] at ([xshift = 1.4em, yshift = 0em]clockextext.east) (rpex3) {};
        \node[text = kanade, anchor = west, inner sep = 0] at ([xshift = 1.9em]clockextext.east) (retargettext) {\tiny Target Recalculation Point};
        \node[text = kanade, anchor = east, inner sep = 0] at ([xshift = -.5em]blockex.west) (epochextext) {\tiny Target Recalculation Epoch};
        \node[draw = kanade, rectangle, fill = kanade!80, minimum size = 1em, inner sep = 0, anchor = east] at ([xshift = -.2em]epochextext.west) (epochex) {};
        \node[text = kanade, anchor = east, inner sep = 0] at ([xshift = -.5em]epochex.west) (intervalextext) {\tiny Clock Synchronization Interval};
        \node[draw = kanade, rectangle, fill = kanade!60, minimum size = 1em, inner sep = 0, anchor = east] at ([xshift = -.2em]intervalextext.west) (intervalex) {};

        \begin{scope}[on background layer]
            \node[fit = (intervalex) (retargettext), draw = kanade, rectangle, inner sep = .4em] (scope1) {};
        \end{scope}
    \end{tikzpicture}

    \caption{An illustration of the clock synchronization protocol execution with one target recalculation epoch consisting of four clock synchronization intervals.}
    \label{fig:protocol-execution}
\end{figure}

\begin{cccItemize}[noitemsep]
    \item It replaces Bitcoin's original clock maintenance solution\footnote{In Bitcoin's original implementation, miners will adjust their time based on three different sources: (i) their local system clock; (ii) the median of clock values from peers; (iii) the human operator (if the first two disagrees).} with a new clock synchronization scheme, which requires parties to use 2-for-1 PoWs~\cite{EC:GarKiaLeo15} to mine and emit clock synchronization beacons and include them in an upcoming block.
    %
    Furthermore, protocol participants will periodically adjust their local clock values based on the beacons collected in the blockchain and their (local) receiving time.

    \item Events are triggered by counting the number of local rounds (which is different from the convention that events in PoW-based blockchains are triggered by the arrival of blocks).
    %
    In other words, the protocol has a clock synchronization \emph{interval} of length \syncLen and a target recalculation \emph{epoch} of length \diffLen that are defined in terms of the number of rounds; in addition, \diffLen is a multiple of \syncLen.
    %
    Both of these values are hardcoded in the protocol.
    %
    More precisely, parties will call the \textsf{SyncProcedure} sub-protocol (see~\cref{protocol:sync-proc}) when their local clock enters round \protocolTime{\interval}{\interval \cdot \syncLen} (this represents the last round in interval \interval; see below for details on the round structure); and for target in the next epoch they will call \textsf{UpdateMiningTarget} (\cref{protocol:update-mining-target}) when their local clock enters \protocolTime{\interval}{\interval \cdot \syncLen}, where $(\interval \mod (\diffLen / \syncLen)) = 0$ (i.e., at the boundary of every $(\diffLen / \syncLen)$ synchronization intervals).
\end{cccItemize}

Next, we present the basic components that are employed in \timekeeper.

\subsection{Timekeeper Timestamps}
\label{subsec:protocol-timestamps}

As opposed to the conventional approach where blocks' timestamps are integer values, timestamps (both blocks' and beacon values) in \timekeeper are represented by a pair of values interval number and round number
%
\( \protocolTime{\interval}{\round} \in \protocolTime{\mathbb{N}^+}{\mathbb{N}^+}. \)
%
Note that (ideally) one synchronization interval would last for \syncLen rounds (i.e., rounds $((i - 1) \cdot \syncLen, i \cdot \syncLen]$ would belong to the $i$-th interval).
%
However, in \timekeeper we let the lower bound be $0$, which means that timestamps with a somewhat small round number are still valid.
%
Specifically, a timestamp $\protocolTime{\interval}{\round}$ is considered valid if and only if it satisfies the predicate
%
\( \mathsf{validTS}(\interval, \round) \triangleq \round \le \interval \cdot \syncLen. \)
%
We note that this new treatment allows for some small distortion at the end of each interval---i.e., the round number of a few blocks at the beginning of the next interval may be smaller than the last block of the previous interval (we call these ``retorted'' timestamps); see~\cref{fig:timestamp-retortion}.

\begin{figure*}[ht]
    \centering
    \tikzstyle{tbblock} = [rectangle, text=white, inner sep = .6em]

    \begin{tikzpicture} \footnotesize
        \node[tbblock, fill=themeColor!70] (block1) {$\langle 1, 90 \rangle$};
        \node[tbblock, themeColor, anchor = east] at ([xshift = -.25cm]block1.west) (block0) {\ldots};
        \node[tbblock, fill=themeColor!70, anchor = west] at ([xshift = .25cm]block1.east) (block2) {$\langle 1, 99 \rangle$};
        \node[tbblock, fill=themeColor!90, anchor = west] at ([xshift = .25cm]block2.east) (block3) {$\langle 2, 95 \rangle$};
        \node[tbblock, fill=themeColor!90, anchor = west] at ([xshift = .25cm]block3.east) (block4) {$\langle 2, 99 \rangle$};
        \node[tbblock, fill=themeColor!90, anchor = west] at ([xshift = .25cm]block4.east) (block5) {$\langle 2, 108 \rangle$};
        \node[tbblock, themeColor, anchor = west] at ([xshift = .25cm]block5.east) (block6) {\ldots};

        \foreach \x [count=\xi from 1] in {0, 1, ..., 5}
            {
                \draw[->, thick, themeColor] (block\xi.west) -- (block\x.east);
            }

        \node[fill = themeColor!70, minimum width = 1em, minimum height = 1em, anchor = west] at ([xshift = 4em, yshift = .75em]block5.east) (note11) {};
        \node[anchor = west] at ([xshift = 5.2em, yshift = .75em]block5.east)  (note12) {Blocks in interval 1};
        \node[fill = themeColor!90, minimum width = 1em, minimum height = 1em, anchor = west] at ([xshift = 4em, yshift = -.75em]block5.east) (note21) {};
        \node[anchor = west] at ([xshift = 5.2em, yshift = -.75em]block5.east)  (note22) {Blocks in interval 2};

        \begin{scope}[on background layer]
            \node[fit = (note11) (note22), draw = themeColor, rectangle, inner sep = .4em] (scope1) {};
        \end{scope}
    \end{tikzpicture}

    \caption{An illustration of a segment of the blockchain with synchronization interval length $\syncLen = 100$. Blocks can have timestamp values equal to blocks in the previous interval.}
    \label{fig:timestamp-retortion}
\end{figure*}

Consider a chain of blocks in \timekeeper.
%
Their timestamps should increase monotonically in terms of their interval number, and the round number in a single interval should also increase monotonically.
%
More specifically, given two timestamps $\protocolTime{\interval_i}{\round_i}, \protocolTime{\interval_j}{\round_j}$ of two blocks $\block_i, \block_j$ respectively, if $\block_i$ is an ancestor block of $\block_j$, they should satisfy the following predicate:
%
\begin{equation*}
    \mathsf{validTSOrder}(\protocolTime{\interval_i}{\round_i}, \protocolTime{\interval_j}{\round_j}) \triangleq
    \left\{
    \begin{aligned}
        \mathsf{validTS}(\interval_i,              & \round_i) \wedge \mathsf{validTS}(\interval_i, \round_i)          \\
        \wedge \big[ (\interval_i \le \interval_j) & \vee (\interval_i = \interval_j \wedge \round_i < \round_j) \big]
    \end{aligned}
    \right\}.
\end{equation*}

Furthermore, we will overload the notation of comparison operators based on the valid order of timestamps.
%
E.g., ``$=$'' will denote that two timestamps are identical, and $\protocolTime{\interval_1}{\round_1} < \protocolTime{\interval_2}{\round_2}$ if and only if $\mathsf{validTSOrder}(\protocolTime{\interval_1}{\round_1}, \protocolTime{\interval_2}{\round_2})$ holds.
%
Other operators $>, \le, \ge, \neq$ are defined similarly.

We also redefine ``$+, -$'' to describe the timestamp that is $k \in \mathbb{N}$ rounds before (resp., after) \protocolTime{\interval}{\round}.
%
Regarding addition, $\protocolTime{\interval}{\round} + k = \protocolTime{\max \{ \interval, \lceil (\round + k) / \syncLen \rceil \}}{\round + k}$.
%
Intuitively, the additive operation simply adds $k$ to \round, and only increments \interval when it is going to become invalid.
%
For subtraction, $\protocolTime{\interval}{\round} - k = \protocolTime{\max \{ 1, \lceil (\round - k) / \syncLen \rceil \}}{\max \{ 1, \round - k \}}$.
%
In other words, regarding the subtraction operation, we only apply the operation on round, and the interval number is derived from the round after calculation.
%
It does not output timestamps that are not ``normally'' belong to an interval.
%
In case we do the subtraction operation for $k \ge \round$, it will return \protocolTime{1}{1}.

\timekeeper's new approach to timestamps raises questions regarding the ``trimming'' of blockchains by counting the number of rounds.
%
Recall that in~\cite{EC:GarKiaLeo15} the notation $\chainPrefix{\chain}{k}$ represents the chain that results from removing the $k$ rightmost blocks.
%
In this paper, we overload this notation to denote the chain that results from removing blocks with timestamps in the last $k$ rounds with respect to the current time.
%
Specifically, for $\chain = \block_1 \block_2 \ldots \block_n$ and local time \protocolTime{\interval}{\round}, $\chainPrefix{\chain}{k} = \block_1 \block_2 \ldots \block_m$ is the longest chain such that $\forall \block \in \chainPrefix{\chain}{k}, \timestamp{\block} < \protocolTime{\interval}{\round} - k$.
%
In other words, $\block_{m + 1}$ is the first block (if it exists) such that $\timestamp{\block_{m + 1}} \ge \protocolTime{\interval}{\round} - k$ holds.

\subsection{2-for-1 Proofs of Work and Synchronization Beacons}
\label{subsec:2-for-1-pow-sync-beacons}

\emph{2-for-1 PoW} is a technique that allows protocols to utilize a single random oracle $H(\cdot)$ to compose two separate PoW sub-procedures involving two distinct and independent random oracles $H_0(\cdot), H_1(\cdot)$.
%
It was first proposed in~\cite{EC:GarKiaLeo15} in order to achieve a better/optimal corruption threshold (from one-third to one-half) for the solution of the traditional consensus problem using a blockchain.

We refer to \cite{EC:GarKiaLeo15} for more details, and here we present a simple implementation with the clock synchronization application in mind.
%
In order to do the 2-for-1 mining, a party \party prepares a composite input $w$ that is a concatenation of two inputs $w_0, w_1$ of two different sub-procedures $S_0, S_1$, respectively.
%
I.e., $w = w_0 \concat w_1$.
%
After selecting a nonce $ctr$, querying the random oracle with $ctr \concat w$ and getting result $u$, \party checks if $u < T$ which implies success in sub-procedure $S_0$; \party also checks if $\stringRev{u} < T$ (where \stringRev{u} denotes the reverse of a bit-string $u$) which indicates success in sub-procedure $S_1$.
%
After successfully generating a PoW for $S_0$ (resp., $S_1$), in order to let parties others check validity, the proof will include the nonce and the entire composite input $ctr \concat w$.
%
Note that sub-procedure $S_0$ (resp., $S_1$) only cares about its corresponding part $w_0$ (resp., $w_1$), and treat the other part as dummy information.

The 2-for-1 PoW technique has several advantages when compared with the straightforward approach that would simply utilize two different random oracles.
%
The most prominent advantage is that it prevents the adversary \adv from concentrating its computational power on one RO and thus gain advantage in the corresponding sub-procedure.

\paragraph{Synchronization beacons.}
%
In addition to the conventional blocks constituting the blockchain, protocol participants in \timekeeper also produce another type of ``tiny'' blocks using 2-for-1 PoWs.
%
We call these blocks \emph{clock synchronization beacons} (``beacons'' for short) since they are used to report parties' local time and synchronize their clocks.

In more detail, one clock synchronization beacon \beacon is a tuple with the following structure:
%
\[ \beacon \triangleq \langle \protocolTime{\interval}{\round}, \party, \eta_\interval, ctr, blockLabel \rangle, \]
%
where \protocolTime{\interval}{\round} is the local time \beacon reports; \party denotes the identity of its miner; $\eta_\interval$ is some fresh randomness in the current interval; $ctr$ represents the nonce of the PoW and $blockLabel$ is the associated block input.
%
Note that \beacon must record the identity of its miner because there might be multiple beacons, mined by different parties, reporting the same timestamp as well as nonce value; otherwise, it would be impossible for the parties to distinguish such beacons.
%
Worse still, other participants would not be able to distinguish the same beacon \beacon when they receive \beacon multiple times.
%
Regarding $\eta_\interval$, it is a string associated with interval \interval for the purpose of preventing the adversary \adv from mining beacons with \emph{future} timestamps.
%
In other words, protocol participants (including \adv) can only learn $\eta_\interval$ after they have (almost) finished interval $\interval - 1$.
%
We present the structure of intervals in detail and how we compute $\eta_\interval$ in~\cref{subsec:intervals-and-sync-procedure} and treat it as a communal bit-string here.
%
We note that parties can learn $\eta_\interval$ from their local chain, and indeed \beacon does not need to include $\eta_\interval$ (\party can prune those beacons that are invalid with $\eta_\interval$ in their local view).
%
We keep $\eta_\interval$ in the description for convenience.

Regarding the structure of a blockchain block \block, we adopt the similar structure as in~\cite{C:GarKiaLeo17} (with the dummy information in the 2-for-1 PoWs):
%
\[ \block \triangleq \langle h, \st, \protocolTime{\interval}{\round}, ctr, txLabel \rangle, \]
%
where $h$ is the reference to the previous block, \st the Merkle root of the block content, \protocolTime{\interval}{\round} its timestamp, $ctr$ the nonce of PoW, and $txLabel$ the bound beacon input.

We are now ready to describe how the parties in \timekeeper do the 2-for-1 PoW mining.
%
The composite input prepared in \timekeeper is different from the trivial instance above, in that the term \protocolTime{\interval}{\round} appears in both blocks and beacons.
%
Hence, simply concatenating two inputs introduces redundant information in the PoW.
%
When a party \party is ready to perform the mining procedure, \party binds the nonce, the blocks' input and beacon input together as
%
\[ \langle ctr, h, \st, \protocolTime{\interval}{\round}, \party, \eta^\chain_\epoch \rangle \]
%
and hand them over to random oracle \funcRO.
%
Let $u$ denote the result from \funcRO.
%
If $u < T$ (i.e., the block query succeeds), \party finds a new block $\block = \langle h, \st, \protocolTime{\interval}{\round}, ctr, txLabel \rangle$ where $txLabel = \langle \party, \eta^\chain_\epoch \rangle$;
%
if $\stringRev{u} < T$ (the beacon query succeeds), \party gets a new beacon $\beacon = \langle \protocolTime{\interval}{\round}, \party, ctr,\allowbreak blockLabel \rangle$, where $blockLabel = \langle h, \st \rangle$.
%
Note that for the sake of presentation, we reorder the content of blocks and beacons so that they are inconsistent with the input to the PoW.

After receiving the result from \funcRO, \party checks if it was able to successfully generate a new block.
%
In addition, \party checks if he successfully produces a beacon but only when \party's local clock stays in the \emph{beacon mining and inclusion} phase.
%
Namely, \party reports a timestamp that satisfies a certain criterion (details in~\cref{subsec:intervals-and-sync-procedure}).

\subsection{Clock Synchronization Intervals and the Synchronization Procedure}
\label{subsec:intervals-and-sync-procedure}

As mentioned earlier, \timekeeper participants will periodically adjust their local clock.
%
We call the time interval between two adjustment points\footnote{The first interval in particular lies between the beginning of the execution and the first time parties adjust their clock.} a \emph{clock synchronization interval} (or ``interval'' for short).
%
Ideally, one interval will last for \syncLen rounds.
%
The actual number of local rounds that parties observe may differ according to the \shift computed in the previous interval (we will show later that the \shift computed in every interval is well-bounded).
%
When party \party's local clock gets to the last round of an interval, it will call \textsf{SyncProcedure} (\cref{protocol:sync-proc}) which adjusts its local clock and gets the fresh randomness to run the next interval.

\subsubsection{Interval Structure}
\label{subsubsec:interval-structure}

\timekeeper divides one interval into three phases: (i) \emph{view convergence}, (ii) \emph{beacon mining and inclusion}, and (iii) \emph{beacon-set convergence}.
%
The phase parties stay in depends on their local clocks.
%
Furthermore, parties will keep track of the (local) arriving time of a synchronization beacon as long as it is online.
%
In this section we describe these three phases as well as the bookkeeping function and explain the design intention behind them.

\paragraph{View convergence.}
%
When a party \party's local clock reports a time \protocolTime{\interval}{\round} such that $\round < (\interval - 1) \cdot \syncLen + \CPLen$, \party is in the \emph{view convergence} phase.
%
Note that this also includes rounds with potentially retorted timestamps.
%
In this phase, if \party is alert, it will try to mine the next block with the 2-for-1 PoW technique (i.e., the input information that \party forwards to the \funcRO functionality does not need to be changed); nonetheless, \party will not check if he successfully mines a beacon after \party acquires the output.
%
This is because all the beacons obtained in this phase are \emph{invalid} in that they report an undesirable timestamp.

The general motivation for introducing the view convergence phase and letting parties wait for some period of time at the beginning of an interval is that we would like parties to start mining beacons with a \emph{consistent} view of the previous interval.
%
Since \CPLen is larger than the common prefix parameter (we will quantify \CPLen later, in~\cref{subsec:protocl-params-conditions}), at the end of the view convergence phase of interval $\interval + 1$, alert parties will have a common view of interval \interval.
%
In other words, they will agree on all the blocks in interval \interval, and the adversary \adv will not be able to apply any changes to these blocks.
%
Hence, alert parties agree on the number of blocks in the previous interval, which decides the mining difficulty within the current interval.
%
(This will used in our new target recalculation function, presented in~\cref{subsec:target-recalculation-function}.)
%
Parties will mine beacons with the same difficulty, and this simplifies the protocol description as well as its analysis.
%
Furthermore, alert parties will compute the same fresh randomness as
%
\begin{equation} \label{eq:compute-freshness}
    \eta_{\interval + 1} \triangleq G(\eta_\interval \concat (\interval + 1) \concat v),
\end{equation}
%
where $v$ is the concatenation of all block hashes in interval \interval.
%
Note that we adopt a different hash function $G(\cdot)$ (as opposed to $H(\cdot)$) to compute the next fresh randomness that is not used in the 2-for-1 PoW , which does not consume any queries to random oracle \funcRO.

Recall that by assumption the adversary \adv has full knowledge of the network, and hence it can learn all honest blocks from the previous interval immediately and manipulate the chain at will for up to a number of rounds bounded by the common prefix parameter, allowing \adv to mine the synchronization beacons before the alert parties start to mine.
%
We call this period where \adv starts ahead of time the \emph{pre-mining} stage.
%
Nonetheless, we will show later that there will be at least one block generated by an alert party near the end of interval \interval, which prevents the adversary from pre-mining for too long a time.

\begin{remark} \label{remark:omit-view-convergence-phase}
    We note that, with some modifications, it is safe to get rid of the view convergence phase.
    %
    The fresh randomness will still need to be extracted from the settled part of the chain, so we will replace it with the randomness generated in previous beacon mining and inclusion phase.
    %
    This can be implemented by modifying some parameters but does not change the protocol framework.
    %
    The main difference lies in how validity of a timestamp beacon is checked.
    %
    At the beginning of a target recalculation epoch (which is also the beginning of an interval), parties may be mining on different chains, and hence mining beacons under different targets.
    %
    In order to check if the beacon target is correctly computed, some additional information (e.g., the block height of the previous target recalculation epoch) should be included in the block header (and the height should not be less than the number of blocks in the settled part).
    %
    Moreover, when parties run the synchronization procedure at the first interval in an epoch, they should use the ``weighted'' median timestamp instead of the plain median (and ``weighted'' means we would assign a weight to each timestamp based on its difficulty).
    %
    Details on applying the median timestamp are given in~\cref{subsubsec:sync-procedure}.
\end{remark}

\paragraph{Beacon mining and inclusion.}
%
When a party \party's local clock is in rounds \protocolTime{\interval}{\round} satisfying $(\interval - 1) \cdot \syncLen + \CPLen \le \round \le \interval \cdot \syncLen - \CPLen$, \party is in the \emph{beacon mining and inclusion} phase.
%
Next, we define the predicate \Isync{\interval} to extract the set of timestamps in this phase.
%
Formally,
%
\begin{equation} \label{eq:isync-def}
    \Isync{\interval} \triangleq \{(\interval - 1) \cdot \syncLen + \CPLen, \ldots, \interval \cdot R - \CPLen \}.
\end{equation}
%
For convenience, we slightly overload this predicate.
%
When the input is a timestamp, \Isync{\protocolTime{\interval}{\round}} outputs whether \protocolTime{\interval}{\round} stays in a beacon mining and inclusion phase.
%
I.e.,  $\Isync{\protocolTime{\interval}{\round}} = \true$ if $\round \in \Isync{\interval}$, and \false otherwise.

After entering this phase, \party will use a 2-for-1 PoW to mine both blocks and clock synchronization beacons.
%
During interval \interval, the output will be a beacon which indicates its local time and value $\beacon \triangleq \langle \protocolTime{\interval}{\round}, \party, ctr, blockLabel \rangle$.
%
Regarding the mining difficulty, \timekeeper will set the same target value for blocks and beacons.\footnote{We will adopt the same target for simplicity. Indeed, maintaining a constant ratio between the difficulty level of blocks and that of beacons will work.}
%
In other words, the expected number of blocks and of beacons in this phase are
equal.

After a beacon is successfully generated, it will be diffused into the network via $\funcDiffuse^{\mathsf{sync}}$.
%
\party will include a beacon \beacon into the pending block content if \beacon is valid w.r.t. the current interval.
%
Next, we describe how they check the validity of a beacon is checked.
%
The format of a beacon \beacon with respect to interval \interval is correct if and only if it reports a timestamp \protocolTime{\interval}{\round} such that $\round \in \Isync{\interval}$.
%
We say a beacon \beacon is \emph{valid w.r.t. chain \chain} if and only if its format is correct and the hash value of \beacon (after concatenating with the fresh randomness in \chain) is smaller than the corresponding mining target.
%
\party will try to include all the (valid) beacons mined in the current interval \interval with timestamps earlier than the current local time but which have not yet been included in the blockchain.
%
Specifically, at round \protocolTime{\interval}{\round}, all valid beacons recording timestamp \protocolTime{\interval}{u} with $u \le \round$ will get into \party's pending block content.

When \party's local clock goes past the last round of beacon mining and inclusion phase, it stops checking the beacon hash output and it no longer includes beacons in the next block. Beacons that are generated and diffused right at the end of this phase get dropped.

\paragraph{Beacon-set convergence.}
%
The third and last phase---\emph{beacon-set convergence}---consists of the last \CPLen rounds in an interval.
%
In other words, a party \party is in this phase when \party reports a timestamp
\protocolTime{\interval}{\round} with $\round > \interval \cdot \syncLen - \CPLen$.
%
During this phase, \party behaves similar to the first phase.
%
I.e., it will not check for the 2-for-1 PoW result to see if the beacon generation succeeds.

Parties have to wait for at least \CPLen rounds to ensure that they share a \emph{consistent} view of the set of beacons included in the current interval (except with some negligible probability).
%
This phase cannot be omitted (as opposed to the case of first phase mentioned in~\cref{remark:omit-view-convergence-phase}) since only when parties agree on the same beacon set can the synchronization procedure maintain the protocol's security properties (\cref{sec:protocol-analysis}).

\paragraph{Beacon arrival booking.}
%
In order to adjust its clock, \party also needs the local receiving time of all beacons that have been included in the chain.
%
Hence, \party will maintain a local registry that records the beacons it receives as well as their arrival time.
%
More specifically, this local beacon ledger is an array of synchronization beacons.
%
For each beacon \beacon, a pair $(a, \mathrm{flag}) \in \protocolTime{\mathbb{N}^+} {\mathbb{N}^+} \times \{\mathsf{final}, \mathsf{temp} \}$ is assigned to it.
%
Consider a round \protocolTime{\interval}{\round} when \party receives a beacon \beacon with $\timestamp{\beacon} = \protocolTime{\interval'}{\round'}$.
%
\begin{cccItemize}[noitemsep]
    \item If $\interval' \le \interval$, which means the beacon \beacon is generated in the current or previous interval.\footnote{Beacons generated in previous intervals are stale in that \party has already passed the synchronization point associated with these beacons, and they will never be used in the future. We list them for completeness.}
    %
    \party will drop \beacon if it is not valid w.r.t. its local chain; otherwise, \party will assign $(\protocolTime{\interval}{\round}, \mathsf{final})$ to \beacon.
    %
    This means that all the information gathering regarding this beacon has been finalized and it is ready to be used.

    \item If $\interval' > \interval$, the beacon is generated in the future.
    %
    \party will assign $(\protocolTime{\interval}{\round}, \allowbreak \mathsf{temp})$ to \beacon, which indicates that modifications on the receiving time may be applied in the future.
    %
    Note that parties may not know the fresh randomness in future intervals (for example, if they are newly joint parties and have not yet synchronized with the blockchain or they are alert but receive forthcoming beacons).
    %
    Hence they cannot check the validity of beacons with $\mathsf{temp}$ flag.
    %
    Nevertheless, invalid beacons would be excluded from the registry after \party learns the upcoming fresh randomness.
\end{cccItemize}
%
If \party receives multiple beacon messages with the same creator and time reported, \party will adopt the first one it receives as its arrival time.

\subsubsection{The Synchronization Procedure}
\label{subsubsec:sync-procedure}

At the end of an interval (i.e., when the local time reports $\round = \interval \cdot \syncLen$), parties will use the beacons information to compute a value \shift that indicates how much the logical clock should be adjusted.
%
(See~\cref{protocol:sync-proc} for the complete specification.)

\paragraph{Adjusting the local clock.}
%
When a party \party's local clock reaches round \protocolTime{\interval}{\interval \cdot \syncLen} and \party has finished the round's regular mining procedure, \party will adjust its local clock based on the beacons recorded on chain and their local receiving time.
%
More specifically, \party will extract all the beacons from the beacon mining and inclusion phase, and compute the differences between their timestamp and local receiving time $\timestamp{\beacon} - \mathsf{arrivalTime}(\beacon)$.
%
Since the timestamp of \beacon and its arrival time share the same interval index, we only need to compute the difference between their round numbers.
%
Subsequently, all the beacons will be ordered based on this difference and a \shift will be computed by selecting the median difference therein.
%
Formally,
%
\begin{equation} \label{eq:sync-shift}
    \shift^{\party}_\interval \triangleq \med \{ \timestamp{\beacon} - \mathsf{arrivalTime}(\beacon) \mathbin| \beacon \in \mathcal{S}^\party_\interval \}.
\end{equation}
%
In case there are two median beacons $\beacon_1, \beacon_2$, parties will adjust
$\shift^{\party}_\interval \triangleq \lceil (\timestamp{\beacon_1} - \mathsf{arrivalTime}(\beacon_1) + \timestamp{\beacon_2} - \mathsf{arrivalTime}(\beacon_2)) / 2 \rceil$.
%
Afterwards, \party will update its local clock to \protocolTime{\interval + 1}{\round + \shift}.
%
Later we show that this update strategy in the synchronization procedure allows parties' clocks to remain in a narrow interval and do not deviate too much from the nominal time.

Note that parties will enter local round \protocolTime{\interval}{\round} where $\round = \interval \cdot \syncLen$ only once.
%
If they enter some time \protocolTime{\interval'}{\round} in the future, we will get $\interval' > \interval$ and they will never revert back.

\paragraph{Mining with backward-set clocks.}
%
After the adjustment at the end of intervals, and \shift is added to \party's local clock, it may set its local time to values \protocolTime{\interval}{\round} such that $\round \le (\interval - 1) \cdot \syncLen$ (i.e., the retortion effect that was mentioned earlier).
%
Nonetheless, \party can continue to mine blocks with this timestamp and its local clock will eventually proceed to a time value of regular format (i.e., $\round > (\interval - 1) \cdot \syncLen$).

We compare this treatment with the similar scenario in a PoS blockchain \cite{EC:BGKRZ21}.
%
In~\cite{EC:BGKRZ21}, setting local clocks backward is never a problem since parties can keep silent during this period.
%
Due to the nature of PoS-based blockchains, parties do not need to do anything if they are not assigned the leader slot.
%
In our context, however, adopting the same `silence' policy contradicts the basic nature of PoW-based blockchains as parties will forfeit the chance to extend their local chain.
%
In other words, there is no point for an activate party to not make RO queries.
%
This is taken care of by \timekeeper's timestamping scheme.

\paragraph{Updating the beacon arrival time registry.}
%
Notice that the beacon information stored in a party \party's arrival time registry is closely related to which interval \party stays in; after \party enters the next interval, it needs to update the beacon bookkeeping.
%
\party will apply a shift computation for all beacons with flag $\mathsf{temp}$.
%
Furthermore, for those beacons that report a timestamp with interval equal to the incoming one, their flag will be set to $\mathsf{final}$.
%
In more detail, at the end of interval \interval, for all eligible \beacon in the beacon registry, their associated pair $(\protocolTime{\interval_\beacon}{\round_\beacon}, \mathsf{temp})$ will be updated to $(\protocolTime{\interval_\beacon}{\round_\beacon + \shift}, \mathsf{final})$ if $\interval_\beacon = \interval + 1$.
%
Note that for those beacons whose flags are set to $\mathsf{final}$, \party will remove all invalid ones from the registry after the update.

\subsection{The Target Recalculation Function}
\label{subsec:target-recalculation-function}

If the mining target is not set appropriately (``appropriately'' means that the block generation rate according to the current hashing power and target is somewhat steady; see~\cite{C:GarKiaLeo17}), PoW-based blockchain protocols fail to maintain any of the security properties in a permissionless environment.
%
In Bitcoin, the target is adjusted after receiving the last block of the current epoch (and an epoch consists of 2016 blocks).
%
Based on the time elapsed to mine these blocks, a new target is set based on the previous target value and the variation is proportional to the time elapsed.
%
Note that Bitcoin's target recalculation function is not the only way to adjust the difficulty level.
%
A large number of other recalculation functions have been proposed in alternate blockchains (e.g., Ethereum, Bitcoin Cash, Litecoin), with their security asserted by either theoretical analysis or empirical data.

In \timekeeper, we propose a new target recalculation function that is suitable for the new setting.
%
Intuitively, our function is a reversed version of Bitcoin's original function, namely, protocol participants wait for some fixed number of rounds \diffLen (in their local view) to update the difficulty level.
%
We call such \diffLen number of rounds a {\em target recalculation epoch}.
%
Moreover, \timekeeper sets \diffLen as a multiple of \syncLen, which makes the target recalculation epoch consist of several clock synchronization intervals, and the start and end point of an epoch coincide with the start and end of different synchronization intervals.
%
Recall that in the \timekeeper timestamp scheme introduced in~\cref{subsec:protocol-timestamps}, the first term in \protocolTime{\interval}{\round} does not directly reflect which target recalculation epoch it is in.
%
For simplicity, we introduce function \textsf{TargetRecalcEpoch} that maps the protocol timestamp to the target recalculation epoch it belongs to:
%
\[ \mathsf{TargetRecalcEpoch}(\protocolTime{\interval}{\round}) \triangleq \lceil \interval / (\diffLen / \syncLen) \rceil. \]
%
In addition, we introduce a function \textsf{EpochBlocks} which extracts all the blocks in chain \chain that belong to target recalculation epoch \epoch.
%
Formally, given $\epoch \ge 1$,
%
\[ \mathsf{EpochBlocks}(\chain, \epoch) \triangleq \{ \block : \block \in \chain \wedge \mathsf{TargetRecalcEpoch}(\timestamp{\block}) = \epoch \}. \]
%
Also for convenience, we let \textsf{EpochBlockCount} be a function that returns the number of blocks in chain \chain that belong to epoch \epoch.
%
We also extend the input domain of epoch numbers to 0 and let it output $\varLambda_{\mathsf{epoch}}$ (the ideal number of blocks) to capture the fact that the target at the beginning of an execution is set appropriately and hence maintains the ideal block generation rate.
%
Formally,
%
\begin{equation} \label{eq:epochblockcount}
    \mathsf{EpochBlockCount}(\chain, \epoch) \triangleq
    \left\{
    \begin{aligned}
         & |\mathsf{EpochBlocks}(\chain, \epoch)| \\
         & \varLambda_{\mathsf{epoch}}
    \end{aligned}
    \right.
    ~~
    \left.
    \begin{aligned}
         & if~\epoch \ge 1 \\
         & if~\epoch = 0
    \end{aligned}
    \right.
\end{equation}

Going back to the algorithm, for the first epoch ($\epoch  = 1$) parties will adopt the target value of the genesis block ($T_0$). I.e., $T_ 1 = T_ 0$.
%
Regarding other epochs ($\epoch > 1$), parties will figure out how many blocks are produced in the previous epoch, and set the next target based on the previous one.
%
This variation is proportional to the ratio of expected number of blocks $\varLambda_{\mathsf{epoch}}$ and the actual number.
%
I.e., for epoch $\epoch + 1$,
%
\begin{equation} \label{eq:next-target}
    T_{\epoch + 1} \triangleq \frac{\varLambda_{\mathsf{epoch}}}{\varLambda} \cdot T_\epoch, ~~ \epoch \in \mathbb{N}^+,
\end{equation}
%
where $\varLambda$ is the number of blocks in epoch \epoch---in other words, the size of $\mathsf{EpochBlocks}(\chain, \epoch)$.

In order to prevent the ``raising difficulty attack'' \cite{EPRINT:Bahack13}, the maximal target variation in a single recalculation step still needs to be bounded (we denote this bound by $\tau$).
%
Specifically, if $\varLambda > \tau \cdot \varLambda_{\mathsf{epoch}}$, $T_{\epoch + 1}$ will be set as $T_\epoch / \tau$; on the other hand, if $\varLambda < \varLambda_{\mathsf{epoch}} / \tau$, $T_{\epoch + 1}$ will be set as $\tau \cdot T_\epoch$.

\begin{remark}
    We observe that, compared to the Bitcoin case, the adversary \adv in \timekeeper is in a much worse position to carry out the raising difficulty attack.
    %
    This is because in Bitcoin, in order to significantly raise the difficulty in the next epoch, \adv only needs to mine 2016 blocks with close timestamps; in the case of \timekeeper, however, the adversary has to mine $\tau \cdot \varLambda_{\mathsf{epoch}}$ blocks (with fake timestamps) in order to raise the same level of difficulty.
    %
    The number of blocks that \adv needs to prepare is $\tau$ times larger than that in Bitcoin (assuming both protocols share the same number of expected blocks in an epoch).
\end{remark}

\subsection{Newly Joining Parties}
\label{subsec:newly-joint-parties}

Recall that \timekeeper runs in a permissionless environment where parties can join and leave at will.
%
As such, it is essential that newly joining parties can learn the protocol time to become alert and participate in the core mining process.
%
More specifically, after the joining procedure, newly joining party \party's local clock should report a time in a sufficiently narrow interval with all other alert parties, at which point \party can claim also being alert.

Based on the fine-grained classification of types of parties in our dynamic participation model (\cref{subsec:dynamic-participation}), newly joining parties can be classified into two types: (i) parties that are temporarily de-registered from \funcRO, and (ii) parties that start with bootstrapping from the genesis block, or parties that temporarily lose the network connection (i.e., de-registered from \funcDiffuse), or parties that are temporarily de-registered from \funcImpClock.

For parties that are stalled for a while, since they do not miss any clock tick or other necessary information from the network, they can easily re-join by calling the procedure \textsf{SimulateClockAdjustments} (\cref{protocol:simulate-clock-adjustment}).
%
For the rest of newly joining parties, they will be classified as de-synchronized (note that parties are aware of their synchronization status), and will run the joining procedure \textsf{JoinProc}, which we now describe.

\paragraph{Procedure \textsf{JoinProcedure}.}
%
In order to synchronize its clock, a newly joining party \party needs to ``listen'' to the protocol for sufficiently long time.
%
We describe the joining process below, which is similar to that in~\cite{EC:BGKRZ21}.
%
The main difference is that we adopt the heaviest-chain selection rule in order to adapt to the PoW context.
%
The complete specification of this protocol is presented in~\cref{protocol:join-procedure}, and the default parameters values are summarized in~\cref{table:join-proc-param}.

\begin{tabularx}{\textwidth}{@{\hskip .15in} c @{\hskip .3in} c @{\hskip .3in} c @{\hskip .15in}}
    \toprule
    \textbf{Parameter}
     & \textbf{Default}
     & \textbf{Phase}
    \\ \midrule
    $t_{\mathsf{off}}$
     & $2\CPLen$
     & B
    \\ \midrule
    $t_{\mathsf{gather}}$
     & $5\syncLen / 2$
     & C
    \\ \midrule
    $t_{\mathsf{pre}}$
     & $3\CPLen$
     & D
    \\ \bottomrule
    \caption{Parameters of the joining procedure and their corresponding phases.}
    \label{table:join-proc-param}
\end{tabularx}

\begin{cccItemize}[nosep]
      \item \textbf{Phase A (state reset).}
      %
      When all resources are available to \party, after resetting all its local variables, \party invokes the main round procedure triggering the join procedure.

      \item \textbf{Phase B (chain convergence, with parameter $t_{\mathsf{off}}$).}
      %
      In the second activation upon a \textsc{maintain-ledger} command, the party will jump to phase B and stay in phase B for $t_{\mathsf{off}}$ rounds.
      %
      During this phase, the party applies the \emph{heaviest-chain selection rule} \textsf{maxvalid} to filter its incoming chains.
      %
      The motivation behind Phase B is to let \party build a chain that shares a sufficiently long common prefix with all alert parties.
      %
      Note that since \party has not yet learnt the protocol time, it cannot filter out chains that should be put aside in the \futureChains.
      %
      Hence, the chain held by \party may still contain a long suffix built entirely by the adversary.
      %
      However, it can be guaranteed (\cref{lemma:goodskew-new-party}(a)) that this adversarial fork can happen for up to $k$ rounds ahead.
      %
      Thus, the beacons recorded before the fork can be used to compute the adjustment and their local arrival times will be reliable.

      \item \textbf{Phase C (beacon gathering, with parameter $t_{\mathsf{gather}}$).}
      %
      Once a party \party has finished Phase B, it continues with Phase C, the beacon-gathering phase.
      %
      During this phase, \party continues to collect and filter chains as in Phase B.
      %
      In addition, \party now processes and bookkeeps the beacons received from $\funcDiffuse^{\mathsf{sync}}$.
      %
      At a high level, this phases' length parameter $t_{\mathsf{gather}}$ guarantees that: (i) enough beacons are recorded to compute a reliable time shift; and (ii) enough time has elapsed so that the blockchain reaches agreement on the set of (valid) beacons to use.
      %
      At the end of Phase C, \party is able to reliably judge valid arrival times.

      \item \textbf{Phase D (shift computation, with parameter $t_{\mathsf{pre}}$).}
      %
      Since party \party has now built a blockchain sharing a common prefix with alert parties, and has bookkeeped synchronization beacons for a sufficiently long time, \party starts from the earliest interval $i^*$ such that (i) the arrival times of all beacons included in blocks within the beacon mining and inclusion phase of interval $i^*$ have been locally bookkeeped; and (2) all of these beacons arrived sufficiently later than the start of Phase C (parameterized by $t_{\mathsf{pre}}$ rounds).
      %
      Based on this information, \party computes the shift value as alert parties do at the boundary of synchronization interval $i^*$.
      %
      \party concludes Phase D when the adjusted time is a valid timestamp in interval $i^* + 1$ (in other words, \round does not exceed $(i^* + 1) \syncLen$); otherwise, \party updates the local arrival time of beacons with flag $\mathsf{temp}$ and repeats the above process with interval $i^* + 1$.
      %
      We note that if Phase D involves the computation w.r.t. multiple intervals, the local time may temporarily be set as an invalid timestamp.
      %
      Nevertheless, eventually after \party has passed $(2\CPLen + 5\syncLen / 2)$ (local) rounds, \party will end up with a valid timestamp that with overwhelming probability is close enough to those of all alert parties.
\end{cccItemize}
