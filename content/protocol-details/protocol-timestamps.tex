\subsection{Timekeeper Timestamps}
\label{subsec:protocol-timestamps}

As opposed to the conventional approach where blocks' timestamps are integer values, timestamps (both blocks' and beacon values) in \timekeeper are represented by a pair of values interval number and round number
%
\( \protocolTime{\interval}{\round} \in \protocolTime{\mathbb{N}^+}{\mathbb{N}^+}. \)
%
Note that (ideally) one synchronization interval would last for \syncLen rounds (i.e., rounds $((i - 1) \cdot \syncLen, i \cdot \syncLen]$ would belong to the $i$-th interval).
%
However, in \timekeeper we let the lower bound be $0$, which means that timestamps with a somewhat small round number are still valid.
%
Specifically, a timestamp $\protocolTime{\interval}{\round}$ is considered valid if and only if it satisfies the predicate
%
\( \mathsf{validTS}(\interval, \round) \triangleq \round \le \interval \cdot \syncLen. \)
%
We note that this new treatment allows for some small distortion at the end of each interval---i.e., the round number of a few blocks at the beginning of the next interval may be smaller than the last block of the previous interval (we call these ``retorted'' timestamps); see~\cref{fig:timestamp-retortion}.

\begin{figure*}[ht]
    \centering
    \tikzstyle{tbblock} = [rectangle, text=white, inner sep = .6em]

    \begin{tikzpicture} \footnotesize
        \node[tbblock, fill=themeColor!70] (block1) {$\langle 1, 90 \rangle$};
        \node[tbblock, themeColor, anchor = east] at ([xshift = -.25cm]block1.west) (block0) {\ldots};
        \node[tbblock, fill=themeColor!70, anchor = west] at ([xshift = .25cm]block1.east) (block2) {$\langle 1, 99 \rangle$};
        \node[tbblock, fill=themeColor!90, anchor = west] at ([xshift = .25cm]block2.east) (block3) {$\langle 2, 95 \rangle$};
        \node[tbblock, fill=themeColor!90, anchor = west] at ([xshift = .25cm]block3.east) (block4) {$\langle 2, 99 \rangle$};
        \node[tbblock, fill=themeColor!90, anchor = west] at ([xshift = .25cm]block4.east) (block5) {$\langle 2, 108 \rangle$};
        \node[tbblock, themeColor, anchor = west] at ([xshift = .25cm]block5.east) (block6) {\ldots};

        \foreach \x [count=\xi from 1] in {0, 1, ..., 5}
            {
                \draw[->, thick, themeColor] (block\xi.west) -- (block\x.east);
            }

        \node[fill = themeColor!70, minimum width = 1em, minimum height = 1em, anchor = west] at ([xshift = 4em, yshift = .75em]block5.east) (note11) {};
        \node[anchor = west] at ([xshift = 5.2em, yshift = .75em]block5.east)  (note12) {Blocks in interval 1};
        \node[fill = themeColor!90, minimum width = 1em, minimum height = 1em, anchor = west] at ([xshift = 4em, yshift = -.75em]block5.east) (note21) {};
        \node[anchor = west] at ([xshift = 5.2em, yshift = -.75em]block5.east)  (note22) {Blocks in interval 2};

        \begin{scope}[on background layer]
            \node[fit = (note11) (note22), draw = themeColor, rectangle, inner sep = .4em] (scope1) {};
        \end{scope}
    \end{tikzpicture}

    \caption{An illustration of a segment of the blockchain with synchronization interval length $\syncLen = 100$. Blocks can have timestamp values equal to blocks in the previous interval.}
    \label{fig:timestamp-retortion}
\end{figure*}

Consider a chain of blocks in \timekeeper.
%
Their timestamps should increase monotonically in terms of their interval number, and the round number in a single interval should also increase monotonically.
%
More specifically, given two timestamps $\protocolTime{\interval_i}{\round_i}, \protocolTime{\interval_j}{\round_j}$ of two blocks $\block_i, \block_j$ respectively, if $\block_i$ is an ancestor block of $\block_j$, they should satisfy the following predicate:
%
\begin{equation*}
    \mathsf{validTSOrder}(\protocolTime{\interval_i}{\round_i}, \protocolTime{\interval_j}{\round_j}) \triangleq
    \left\{
    \begin{aligned}
        \mathsf{validTS}(\interval_i,              & \round_i) \wedge \mathsf{validTS}(\interval_i, \round_i)          \\
        \wedge \big[ (\interval_i \le \interval_j) & \vee (\interval_i = \interval_j \wedge \round_i < \round_j) \big]
    \end{aligned}
    \right\}.
\end{equation*}

Furthermore, we will overload the notation of comparison operators based on the valid order of timestamps.
%
E.g., ``$=$'' will denote that two timestamps are identical, and $\protocolTime{\interval_1}{\round_1} < \protocolTime{\interval_2}{\round_2}$ if and only if $\mathsf{validTSOrder}(\protocolTime{\interval_1}{\round_1}, \protocolTime{\interval_2}{\round_2})$ holds.
%
Other operators $>, \le, \ge, \neq$ are defined similarly.

We also redefine ``$+, -$'' to describe the timestamp that is $k \in \mathbb{N}$ rounds before (resp., after) \protocolTime{\interval}{\round}.
%
Regarding addition, $\protocolTime{\interval}{\round} + k = \protocolTime{\max \{ \interval, \lceil (\round + k) / \syncLen \rceil \}}{\round + k}$.
%
Intuitively, the additive operation simply adds $k$ to \round, and only increments \interval when it is going to become invalid.
%
For subtraction, $\protocolTime{\interval}{\round} - k = \protocolTime{\max \{ 1, \lceil (\round - k) / \syncLen \rceil \}}{\max \{ 1, \round - k \}}$.
%
In other words, regarding the subtraction operation, we only apply the operation on round, and the interval number is derived from the round after calculation.
%
It does not output timestamps that are not ``normally'' belong to an interval.
%
In case we do the subtraction operation for $k \ge \round$, it will return \protocolTime{1}{1}.

\timekeeper's new approach to timestamps raises questions regarding the ``trimming'' of blockchains by counting the number of rounds.
%
Recall that in~\cite{EC:GarKiaLeo15} the notation $\chainPrefix{\chain}{k}$ represents the chain that results from removing the $k$ rightmost blocks.
%
In this paper, we overload this notation to denote the chain that results from removing blocks with timestamps in the last $k$ rounds with respect to the current time.
%
Specifically, for $\chain = \block_1 \block_2 \ldots \block_n$ and local time \protocolTime{\interval}{\round}, $\chainPrefix{\chain}{k} = \block_1 \block_2 \ldots \block_m$ is the longest chain such that $\forall \block \in \chainPrefix{\chain}{k}, \timestamp{\block} < \protocolTime{\interval}{\round} - k$.
%
In other words, $\block_{m + 1}$ is the first block (if it exists) such that $\timestamp{\block_{m + 1}} \ge \protocolTime{\interval}{\round} - k$ holds.
