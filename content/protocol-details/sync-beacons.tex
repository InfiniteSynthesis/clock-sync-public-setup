\subsection{2-for-1 Proofs of Work and Synchronization Beacons}
\label{subsec:2-for-1-pow-sync-beacons}

\emph{2-for-1 PoW} is a technique that allows protocols to utilize a single random oracle $H(\cdot)$ to compose two separate PoW sub-procedures involving two distinct and independent random oracles $H_0(\cdot), H_1(\cdot)$.
%
It was first proposed in~\cite{EC:GarKiaLeo15} in order to achieve a better/optimal corruption threshold (from one-third to one-half) for the solution of the traditional consensus problem using a blockchain.

We refer to \cite{EC:GarKiaLeo15} for more details, and here we present a simple implementation with the clock synchronization application in mind.
%
In order to do the 2-for-1 mining, a party \party prepares a composite input $w$ that is a concatenation of two inputs $w_0, w_1$ of two different sub-procedures $S_0, S_1$, respectively.
%
I.e., $w = w_0 \concat w_1$.
%
After selecting a nonce $ctr$, querying the random oracle with $ctr \concat w$ and getting result $u$, \party checks if $u < T$ which implies success in sub-procedure $S_0$; \party also checks if $\stringRev{u} < T$ (where \stringRev{u} denotes the reverse of a bit-string $u$) which indicates success in sub-procedure $S_1$.
%
After successfully generating a PoW for $S_0$ (resp., $S_1$), in order to let parties others check validity, the proof will include the nonce and the entire composite input $ctr \concat w$.
%
Note that sub-procedure $S_0$ (resp., $S_1$) only cares about its corresponding part $w_0$ (resp., $w_1$), and treat the other part as dummy information.

The 2-for-1 PoW technique has several advantages when compared with the straightforward approach that would simply utilize two different random oracles.
%
The most prominent advantage is that it prevents the adversary \adv from concentrating its computational power on one RO and thus gain advantage in the corresponding sub-procedure.

\paragraph{Synchronization beacons.}
%
In addition to the conventional blocks constituting the blockchain, protocol participants in \timekeeper also produce another type of ``tiny'' blocks using 2-for-1 PoWs.
%
We call these blocks \emph{clock synchronization beacons} (``beacons'' for short) since they are used to report parties' local time and synchronize their clocks.

In more detail, one clock synchronization beacon \beacon is a tuple with the following structure:
%
\[ \beacon \triangleq \langle \protocolTime{\interval}{\round}, \party, \eta_\interval, ctr, blockLabel \rangle, \]
%
where \protocolTime{\interval}{\round} is the local time \beacon reports; \party denotes the identity of its miner; $\eta_\interval$ is some fresh randomness in the current interval; $ctr$ represents the nonce of the PoW and $blockLabel$ is the associated block input.
%
Note that \beacon must record the identity of its miner because there might be multiple beacons, mined by different parties, reporting the same timestamp as well as nonce value; otherwise, it would be impossible for the parties to distinguish such beacons.
%
Worse still, other participants would not be able to distinguish the same beacon \beacon when they receive \beacon multiple times.
%
Regarding $\eta_\interval$, it is a string associated with interval \interval for the purpose of preventing the adversary \adv from mining beacons with \emph{future} timestamps.
%
In other words, protocol participants (including \adv) can only learn $\eta_\interval$ after they have (almost) finished interval $\interval - 1$.
%
We present the structure of intervals in detail and how we compute $\eta_\interval$ in~\cref{subsec:intervals-and-sync-procedure} and treat it as a communal bit-string here.
%
We note that parties can learn $\eta_\interval$ from their local chain, and indeed \beacon does not need to include $\eta_\interval$ (\party can prune those beacons that are invalid with $\eta_\interval$ in their local view).
%
We keep $\eta_\interval$ in the description for convenience.

Regarding the structure of a blockchain block \block, we adopt the similar structure as in~\cite{C:GarKiaLeo17} (with the dummy information in the 2-for-1 PoWs):
%
\[ \block \triangleq \langle h, \st, \protocolTime{\interval}{\round}, ctr, txLabel \rangle, \]
%
where $h$ is the reference to the previous block, \st the Merkle root of the block content, \protocolTime{\interval}{\round} its timestamp, $ctr$ the nonce of PoW, and $txLabel$ the bound beacon input.

We are now ready to describe how the parties in \timekeeper do the 2-for-1 PoW mining.
%
The composite input prepared in \timekeeper is different from the trivial instance above, in that the term \protocolTime{\interval}{\round} appears in both blocks and beacons.
%
Hence, simply concatenating two inputs introduces redundant information in the PoW.
%
When a party \party is ready to perform the mining procedure, \party binds the nonce, the blocks' input and beacon input together as
%
\[ \langle ctr, h, \st, \protocolTime{\interval}{\round}, \party, \eta^\chain_\epoch \rangle \]
%
and hand them over to random oracle \funcRO.
%
Let $u$ denote the result from \funcRO.
%
If $u < T$ (i.e., the block query succeeds), \party finds a new block $\block = \langle h, \st, \protocolTime{\interval}{\round}, ctr, txLabel \rangle$ where $txLabel = \langle \party, \eta^\chain_\epoch \rangle$;
%
if $\stringRev{u} < T$ (the beacon query succeeds), \party gets a new beacon $\beacon = \langle \protocolTime{\interval}{\round}, \party, ctr,\allowbreak blockLabel \rangle$, where $blockLabel = \langle h, \st \rangle$.
%
Note that for the sake of presentation, we reorder the content of blocks and beacons so that they are inconsistent with the input to the PoW.

After receiving the result from \funcRO, \party checks if it was able to successfully generate a new block.
%
In addition, \party checks if he successfully produces a beacon but only when \party's local clock stays in the \emph{beacon mining and inclusion} phase.
%
Namely, \party reports a timestamp that satisfies a certain criterion (details in~\cref{subsec:intervals-and-sync-procedure}).
